q/g tagging studies go here.

Start adding outline/discussion of theoretical understanding

\subsection{QJets Volatility and $\ptd$ ($\C{1}{\beta=0}$)}

Simple explanation of correlation, or why does combining volatility and $\ptd$ improve quark versus gluon discrimination.  $\ptd$ ($\C{1}{\beta=0}$) takes small (large) values for a jet with near-democratic energy sharing between particles and large (small) values when the energy of the jet is contained in a few particles.  Because we expect gluons to radiate more particles, we expect that $\ptd_g<\ptd_q$ (or ${\C{1}{\beta=0}}_g>{\C{1}{\beta=0}}_q$).  Now, we expect the volatility of gluon jets to be in general smaller than that of quark jets because there is a greater probability (by a factor of about $C_A/C_F=9/4$) that there was a relatively hard emission in a jet that is not groomed away.  By measuring both volatility and $\ptd$, we are sensitive to both regions of phase space: where a relatively hard emission dominates the mass of the jet as well as the region where many soft emissions set the jet mass.

{\it The following is Steve's discussion of volatility difference between quarks and gluons:}

   Here is the (qualitative) thinking:  typical QCD jet mass distributions look as illustrated on slide 17, although you should really be thinking in terms of plot versus $m/p_T$, since $p_T$ is what sets the scale in the plot.  Qualitatively there is a (very) large peak for $m/p_T \lesssim 0.1$ and you should think of these jets as having masses that arise from multiple soft emissions, some of which are at substantial angles.  It is these components of the jet that are operated on by pruning (reducing the mass dramatically) and that yield the large volatility tail for QCD jets.  For larger $m/p_T$ values there is typically a �shoulder� (my description is clearest on a semi-log plot) that runs out to about $m/pT \sim 0.4 � 0.5$ (where the distribution decreases rapidly).  These are the QCD jets (a small fraction of the total in a given $p_T$ bin) that contain a �hard�, relatively large angle emission, which supplies the bulk of the jet mass.  Such jets are effected only slightly by pruning and should exhibit much smaller volatility than the jets in the (smaller mass) peak region.
 
  With that picture in mind and recalling that the size of the �shoulder� is given by low order perturbation theory (the probability of the one hard emission), we expect that the �shoulder� will be higher for gluons than for quarks (essentially by the usual $C_A/C_F$ color charge factor), as suggested by the lower right plot on slide 17.  Since the shoulder presumably plays a more important role for gluons (since it is larger), one would expect that the volatility distribution for gluons is narrower than quarks, as suggested in the upper left plot on slide 17.  Am I making sense?
 
  On the other hand, the volatility distribution plot indicates that the Q vs G distributions for your cuts are not really very different, which is presumably why it is not a very good discriminant by itself.  But I expect this to depend it detail on where we are operating on the m/pT distributions.  This leads to my request above.  Your $p_T$ bin is pretty broad and I don�t expect the q and g samples to have the same shape within the bin.  Of course, this may not be an issue, but I would like to check.


\subsection{Comparison of Groomed Jet Masses}