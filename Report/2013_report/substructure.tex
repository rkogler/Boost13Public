Describe the specific substructure variables and tagging approaches
that we will be using in this report e.g. n-subjettiness, Q-jets, HTT,
JH tagger. Give the nomenclature that we will use to refer to these
variables/taggers in the rest of the report.\\

\noindent {\bf Johns Hopkins Tagger:} Re-cluster the jet using the Cambridge-Aachen algorithm. The jet is iteratively de-clustered, and at each step the softer prong is discarded if its $p_{\rm T}$ is less than $\delta_p\,p_{\mathrm{T\,jet}}$. This continues until both prongs are harder than the $p_{\rm T}$ threshold, both prongs are softer than the $p_{\rm T}$ threshold, or if they are too close ($|\Delta\eta_{ij}|+|\Delta\phi_{ij}|<\delta_R$); the jet is rejected if either of the latter conditions apply. If both are harder than the $p_{\rm T}$ threshold, the same procedure is applied to each: this results in 2, 3, or 4 subjets. If there exist 3 or 4 subjets, then the jet is accepted: the top candidate is the sum of the subjets, and $W$ candidate is the pair of subjets closest to the $W$ mass. The output of the tagger is $m_t$, $m_W$, and $\theta_{\rm h}$, a helicity angle defined as the angle, measured in the rest frame of the $W$ candidate, between the top direction and one of the $W$ decay products.\\

\noindent {\bf HEPTopTagger:} Re-cluster the jet using the Cambridge-Aachen algorithm. The jet is iteratively de-clustered, and at each step the softer prong is discarded if $m_1/m_{12}>\mu$ (there is not a significant mass drop). Otherwise, both prongs are kept. This continues until a prong has a mass $m_i < m$, at which point it is added to the list of subjets. Filter the jet using $R_{\rm filt}=\mathrm{min}(0.3,\Delta R_{ij})$ (where $\Delta R_{ij}$ is the distance between the two hardest subjets). Select the three subjets whose invariant mass is closest to $m_t$. The output of the tagger is $m_t$, $m_W$, and $\theta_{\rm h}$, a helicity angle defined as the angle, measured in the rest frame of the $W$ candidate, between the top direction and one of the $W$ decay products.\\

\noindent {\bf Trimming:} Re-cluster the jet using the $k_{\rm T}$ algorithm and radius $R_{\rm trim}$. Discard all subjets with $p_{\rm T\,sj}/p_{\rm T\,jet} < f_{\rm cut}$. A $W$ candidate is reconstructed as follows: if there are two subjets, the highest-mass subjet is the $W$ candidate; if there are three subjets, the two subjets with the smallest invariant mass comprise the $W$ candidate. In the case of only one subjet, no $W$ is reconstructed.\\

\noindent {\bf Pruning:} Iteratively de-cluster the jet. At each step, discard the softer branch if $\mathrm{min}(p_{\rm T1},p_{\rm T2}) / p_{\rm T12} < z_{\rm cut}$ and $\Delta R_{12}>D_{\rm cut}$. The subjets are found by de-clustering the pruned jet by up to three splittings. A $W$ candidate is reconstructed as follows: if there are two subjets, the highest-mass subjet is the $W$ candidate; if there are three subjets, the two subjets with the smallest invariant mass comprise the $W$ candidate. In the case of only one subjet, no $W$ is reconstructed.