The characteristic feature of collisions at the LHC is a center-of-mass energy, 7~\tev{} in 2010 and 2011, 
of 8~\tev{} in 2012, and near 14~\tev{} with the start of the second phase of operation in 2015, that is large
compared to even the heaviest of the known particles.  Thus these particles (and also previously unknown ones)
will often be produced at the LHC with
substantial boosts.  As a result, when decaying hadronically, these particles will not be observed as multiple jets in the detector, but rather
as a single hadronic jet with distinctive internal substructure.  This realization has led to a new era of sophistication
in our understanding of  both standard QCD jets and jets containing the decay of a heavy particle, with an array
of new jet observables and detection techniques introduced and studies.  To allow the efficient sharing of 
results from these jet substructure studies a series of BOOST Workshops have been held on a yearly basis:
SLAC (2009, ~\cite{boost09}), 
Oxford University (2010,~\cite{boost10}), Princeton 
University University (2011,~\cite{boost11}),  IFIC Valencia (2012~\cite{boost12}), 
University of Arizona (2013~\cite{boost13}), and, most recently, University College London (2014~\cite{boost2014}).
After each of these meetings Working Groups have functioned during the following year to generate reports
highlighting the most interesting new results, including studies of ever maturing details.   Previous BOOST reports
can be found at \cite{Abdesselam:2010pt,Altheimer:2012mn,Altheimer:2013yza}.

The following report from BOOST 2013 thus views the study and implementation of jet substructure techniques as a fairly
mature field. The report attempts to focus on the question of the correlations between the plethora of observables that have been developed 
and employed, and their dependence on the underlying jet parameters, especially the jet radius $R$ and jet $p_T$.
The report is organized as follows:  NEED TO GENERATE AN OUTLINE OF THE REPORT - ESPECIALLY AS I UNDERSTAND IT
MYSELF.