The characteristic feature of collisions at the LHC is a center-of-mass energy, 7~\tev{} in 2010 and 2011, 
of 8~\tev{} in 2012, and 13\tev{} with the start of the second phase of operation in 2015, that is large
compared to even the heaviest of the known particles.  Thus these particles (and also previously unknown ones)
will often be produced at the LHC with
substantial boosts.  As a result, when decaying hadronically, these particles will not be observed as multiple jets in the detector, but rather
as a single hadronic jet with distinctive internal substructure.  This realization has led to a new era of sophistication
in our understanding of  both standard QCD jets and jets containing the decay of a heavy particle, with an array
of new jet observables and detection techniques introduced and studies.  To allow the efficient sharing of 
results from these jet substructure studies a series of BOOST Workshops have been held on a yearly basis:
SLAC (2009, ~\cite{Boost:2009xx}), 
Oxford University (2010,~\cite{Boost:2010xx}), Princeton 
University University (2011,~\cite{Boost:2011xx}),  IFIC Valencia (2012~\cite{Boost:2012xx}), 
University of Arizona (2013~\cite{Boost:2013xx}), and, most recently, University College London (2014~\cite{Boost:2014xx}).
After each of these meetings Working Groups have functioned during the following year to generate reports
highlighting the most interesting new results, including studies of ever maturing details.   Previous BOOST reports
can be found at \cite{Abdesselam:2010pt,Altheimer:2012mn,Altheimer:2013yza}.

This report from BOOST 2013 thus views the study and implementation of jet substructure techniques as a fairly
mature field, and focuses on the question of the correlations between the plethora of observables that have been developed 
and employed, and their dependence on the underlying jet parameters, especially the jet radius $R$ and jet $p_T$. Samples of quark-, gluon-, W- and Top-initiated jets are reconstructed at the particle-level using \textsc{FastJet} \cite{Cacciari:2011ma}, and the performance, in terms of separating signal from background, of various groomed jet masses and jet substructure observables investigated through Receiver Operating Characteristic (ROC) curves, which show the efficiency to ``tag'' the signal as a function of the efficiency (or rejection, being 1/efficiency) to ``tag'' the background. In new analyses developed for the report, we investigate the separation of a quark signal from a gluon background (q/g tagging), a W signal from a gluon background (W-tagging) and a Top signal from a mixed quark/gluon QCD background (Top-tagging). In the case of Top-tagging, we also investigate the performance of dedicated Top-tagging algorithms, the HepTopTagger \cite{Plehn:2010st} and the Johns Hopkins Tagger \cite{Kaplan:2008ie}. Using multivariate techniques, we study the degree to which the discriminatory information provided by the observables and taggers overlaps, by examining in particular the extent to which the signal-background separation performance increases when two or more variables/taggers are combined, via a Boosted Decision Tree (BDT), into a single discriminant. Where possible, we provide a discussion of the physics behind the structure of the correlations and the $p_{T}$ and $R$ scaling that we observe. 

We present the performance of and correlations between observables in idealized simulations without pile-up and detector resolution effects. 
This allows us to understand the optimal performance of various taggers and jet substructure techniques, and techniques for mitigation of pile-up and detector effects should be designed to recover as much as possible these performance benchmarks. In particular, the relationship between substructure observables, their correlations, and how this depends the jet radius $R$ and jet $p_T$, should not be too sensitive to such effects, and performing these studies in idealized simulations allows us to more clearly elucidate the underlying physics behind our observations. A full study of the performance of pile-up and detector mitigation strategies is beyond the scope of the current report, and will be the focus of upcoming studies.


The report is organized as follows. In Section~\ref{sec:samples} we describe the generation of the Monte Carlo event samples that we use in the studies that follow. In Section~\ref{sec:algssubstructure} we detail the jet algorithms, observables and taggers investigated in each section of the report, and in Section~\ref{sec:multivariate} the multivariate techniques used to combine the one or more of the observables into single discriminants. In Section~\ref{sec:qgtagging} we describe the q/g-tagging studies, in Section~\ref{sec:wtagging} we describe the W-tagging studies, and in Section~\ref{sec:toptagging} we describe the Top-tagging studies. Finally we offer some summary of the studies and general conclusions in Section~\ref{sec:conclusions}.\\

\emph{This report presents original analyses and discussions pertaining to the performance of and correlations between various jet substructure techniques applied to quark/gluon discrimination, $W$-boson tagging, and top tagging. The principal organizers of and contributors to the analyses presented in the report are:~B.~Cooper, S.~D.~Ellis, M.~Freytsis, A.~Hornig, A.~Larkoski, D.~Lopez Mateos, B.~Shuve, and N.~V.~Tran.}