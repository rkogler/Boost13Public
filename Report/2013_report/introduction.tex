The center-of-mass energies at the Large Hadron Collider are large compared to the heaviest of known particles, even after accounting for parton density functions. With the start of the second phase of operation in 2015, the center-of-mass energy will further increase from 7~\tev{} in 2010-2011 and 8~\tev{} in 2012 to 13~\tev{}. Thus, even the heaviest states in the Standard Model (and potentially previously unknown particles) will often be 
produced at the LHC with
substantial boosts, leading to a collimation of the decay products.  For fully hadronic decays, these heavy particles will not be reconstructed as several jets in the detector, but rather
as a single hadronic jet with distinctive internal substructure.  This realization has led to a new era of sophistication
in our understanding of both standard Quantum Chromodynamics (QCD) jets, as well as jets containing the decay of a heavy particle, with an array
of new jet observables and detection techniques introduced and studied to distinguish the two types of jets.  To allow the efficient propagation of 
results from these studies of jet substructure, a series of BOOST Workshops have been held on an annual basis:~SLAC (2009, ~\cite{Boost:2009xx}), 
Oxford University (2010,~\cite{Boost:2010xx}), Princeton 
University  (2011,~\cite{Boost:2011xx}), IFIC Valencia (2012~\cite{Boost:2012xx}), 
University of Arizona (2013~\cite{Boost:2013xx}), and, most recently, University College London (2014~\cite{Boost:2014xx}).
Following each of these meetings, working groups have generated reports
%highlighting the most interesting new results, incuding studies of increasingly fine details. Previous BOOST reports
highlighting the most interesting new results, and often including original particle-level studies. Previous BOOST reports
can be found at \cite{Abdesselam:2010pt,Altheimer:2012mn,Altheimer:2013yza}.

This report from BOOST 2013 thus views the study and implementation of jet substructure techniques as a fairly
mature field, and focuses on the question of the correlations between the plethora of observables that have been developed 
and employed, and their dependence on the underlying jet parameters, especially the jet radius $R$ and jet $p_T$. 
In new analyses developed for the report, we investigate the separation of a quark signal from a gluon background (q/g tagging), a W signal from a gluon background (W-tagging) and a Top signal from a mixed quark/gluon QCD background (Top-tagging). In the case of Top-tagging, we also investigate the performance of dedicated Top-tagging algorithms, the HepTopTagger \cite{Plehn:2010st} and the Johns Hopkins Tagger \cite{Kaplan:2008ie}. We study the degree to which the discriminatory information provided by the observables and taggers overlaps by examining the extent to which the signal-background separation performance increases when two or more variables/taggers are combined in a multivariate analysis. Where possible, we provide a discussion of the physics behind the structure of the correlations and the $p_{T}$ and $R$ scaling that we observe. 


%Samples of quark-, gluon-, W- and Top-initiated jets are reconstructed at the particle-level using \textsc{FastJet} \cite{Cacciari:2011ma}, and the performance, in terms of separating signal from background, of various groomed jet masses and jet substructure observables investigated through Receiver Operating Characteristic (ROC) curves, which show the efficiency to ``tag'' the signal as a function of the efficiency (or rejection, being 1/efficiency) to ``tag'' the background. 

We present the performance of observables in idealized simulations without pile-up and detector resolution effects;
%, with the primary goal of studying the correlations between observables and the dependence on jet radius and $p_{T}$.
the relationship between substructure observables, their correlations, and how these depend on the jet radius $R$ and jet $p_T$ should not be too sensitive to such effects. Conducting  studies using idealized simulations allows us to more clearly elucidate the underlying physics behind the observed performance, and also provides benchmarks for the development of techniques to mitigate pile-up and detector effects. A full study of the performance of pile-up and detector mitigation strategies is beyond the scope of the current report, and will be the focus of upcoming studies.

The report is organized as follows:~in Sections~\ref{sec:samples}-\ref{sec:multivariate}, we describe the methods used in carrying out our analysis, with a description of the Monte Carlo event sample generation in Section~\ref{sec:samples}, the jet algorithms, observables and taggers investigated in our report in Section~\ref{sec:algssubstructure}, and an overview of the multivariate techniques used to combine multiple observables into single discriminants in Section~\ref{sec:multivariate}. Our results follow in Sections~\ref{sec:qgtagging}-\ref{sec:toptagging}, with q/g-tagging studies in Section~\ref{sec:qgtagging}, W-tagging studies in Section~\ref{sec:wtagging}, and Top-tagging studies in Section~\ref{sec:toptagging}. Finally we offer some summary of the studies and general conclusions in Section~\ref{sec:conclusions}.\\

%The report is organized as follows:~in Section~\ref{sec:samples} we describe the generation of the Monte Carlo event samples that we use in the studies that follow. In Section~\ref{sec:algssubstructure} we detail the jet algorithms, observables and taggers investigated in each section of the report, and in Section~\ref{sec:multivariate} the multivariate techniques used to combine the one or more of the observables into single discriminants. In Section~\ref{sec:qgtagging} we describe the q/g-tagging studies, in Section~\ref{sec:wtagging} we describe the W-tagging studies, and in Section~\ref{sec:toptagging} we describe the Top-tagging studies. Finally we offer some summary of the studies and general conclusions in Section~\ref{sec:conclusions}.\\

%\emph{This report presents original analyses and discussions pertaining to the performance of and correlations between various jet substructure techniques applied to quark/gluon discrimination, $W$-boson tagging, and Top tagging. The principal organizers of and contributors to the analyses presented in the report are:~B.~Cooper, S.~D.~Ellis, M.~Freytsis, A.~Hornig, A.~Larkoski, D.~Lopez Mateos, B.~Shuve, and N.~V.~Tran.}

\emph{The principal organizers of and contributors to the analyses presented in this report are:~B.~Cooper, S.~D.~Ellis, M.~Freytsis, A.~Hornig, A.~Larkoski, D.~Lopez Mateos, B.~Shuve, and N.~V.~Tran.}
