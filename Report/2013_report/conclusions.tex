Furthering our understanding of jet substructure is crucial to enhancing the prospects for the discovery of new physical processes at Run II of the LHC. In this report we have studied the performance of jet substructure techniques over a wide range of kinematic regimes that will be encountered in Run II of the LHC. The performance of observables and their correlations have been studied by combining the variables into Boosted Decision Tree (BDT) discriminants, and comparing the background rejection power of this discriminant to the rejection power achieved by the individual variables. The performance of ``all variables'' BDT discriminants has also been investigated, to understand the potential of the ``ultimate'' tagger where ``all'' available particle-level information (at least, all of that provided by the variables considered) is used.

We focused on the discrimination of quark jets from gluon jets, and the discrimination of boosted $W$ bosons and top quarks from the QCD backgrounds. For each, we have identified the best-performing jet substructure observables at particle level, both individually and in combination with other observables. In doing so, we have also provided a physical picture of why certain sets of observables are (un)correlated. Additionally, we have investigated how the performance of jet substructure observables varies with $R$ and \pt, identifying observables that are particularly robust against or susceptible to these changes. In the case of $q/g$ tagging, it seems that the ideal performance can be nearly achieved by combining the most powerful discriminant, the number of constituents of a jet, with just one other variable, $C_1^{\beta =1}$ (or $\tau_1^{\beta=1}$). Many of the other variables considered are highly correlated and provide little additional discrimination. For both top and $W$ tagging, the groomed mass is a very important discriminating variable, but one that can be substantially improved in combination with other variables. There is clearly a rich and complex relationship between the variables considered for $W$ and top tagging, and the performance and correlations between these variables can change considerably with changing jet \pt and $R$. In the case of $W$ tagging, even after combining groomed mass with two other substructure observables, we are still some way short of the ultimate tagger performance, indicating the complexity of the information available, and the complementarity between the observables considered. In the case of top tagging, we have shown that the performance of both the John Hopkins and HEPTopTagger can be improved when their outputs are combined with substructure observables such as $\tau_{32}$ and $C_{3}$, and that the performance of a discriminant built from groomed mass information plus substructure observables is very comparable to the performance of the taggers.  We have optimized the top taggers for particular values of \pt, $R$, and signal efficiency, and studied their performance at other working points. We have found that the performance of observables remains within at most a factor of two of the optimized value, suggesting that the performance of jet substructure observables is not significantly degraded when tagger parameters are only optimized for a few select benchmark points.

In all of $q/g$, $W$ and top tagging, we have observed that the tagging performance improves with increasing \pt. However, whereas for $q/g$ and top tagging the performance improves with decreasing $R$ (for the range of $R$ considered here), the dependence on $R$ for $W$ tagging is more complex, with a peak performance at $R=0.8$ for each \pt bin considered. 

Our analyses were performed with ideal detector and pile-up conditions in order to most clearly elucidate the underlying physical scaling with \pt and $R$. At higher boosts, detector resolution effects will become more important, and with the higher pile-up expected at Run II of the LHC, pile-up mitigation will be crucial for future jet substructure studies. Future studies will be needed to determine which of the observables we have studied are most robust against pile-up and detector effects, and our analyses suggest particularly useful combinations of observables to consider in such studies. 

At the new energy frontier of Run II of the LHC, boosted jet substructure techniques will be more central to our searches for new physics than ever before. By achieving a deeper understanding of the underlying structure of quark, gluon, $W$ and top-initiated jets, as well as the relations between observables sensitive to their respective structures, it is hoped that more sophisticated analyses can be performed that will maximally extend the reach for new physics.\\

{\bf Acknowledgements:}~We thank the University of Arizona for hosting and providing support for the BOOST 2013 workshop, and the US Department of Energy for their support of the workshop.  We also thank Prof. J. Boelts of the University of Arizona School of Art VisCom program and his Fall 2012 ART 465 class for organizing the design competion for the workshop 
poster. In particular, we thank the winner of the competion, Ms. Hallie Bolonkin, for creating the final design.
