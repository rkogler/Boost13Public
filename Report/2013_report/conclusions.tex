In this report, we have studied the discriminatory information in various jet substructure observables/taggers, the correlations between different observables, and how performance and correlation vary as functions of the parameters of the jets, such as their \pt and radius. Furthering our understanding of jet substructure is crucial to improving our understanding of QCD and enhancing the prospects for the discovery of new physical processes at Run II of the LHC. We have studied the performance of jet substructure techniques over a wide range of kinematic regimes that will be encountered in Run II of the LHC. The performance of observables and their correlations have been studied by combining the variables into BDT discriminants, and comparing the background rejection power of this discriminant to the rejection power achieved by the individual variables. The performance of ``all variables'' BDT discriminants has also been investigated, to understand the potential of the ``ultimate'' tagger where ``all'' available information (at least, all of that provided by the variables considered) is used.

We focused on the discrimination of quark jets from gluon jets, and the discrimination of boosted W bosons and top quarks from the QCD backgrounds. For each, we have identified the best-performing jet substructure observables, both individually and in combination with other observables. In doing so, we have also provided a physical picture of why certain sets of observables are (un)correlated. Additionally, we have investigated how the performance of jet substructure observables varies with $R$ and \pt, identifying observables that are particularly robust against or susceptible to these changes. Finally, in the case of boosted top quarks, we have optimized taggers for a particular value of \pt, $R$, and signal efficiency, and studied their performance at other working points. We have found that the performance of observables remains within a factor of two of the optimized value, suggesting that the performance of jet substructure observables is not significantly degraded when tagger parameters are only optimized for a few select benchmark points.

Our analyses were performed with ideal detector and pile-up conditions in order to most clearly elucidate the underlying physical scaling with \pt and $R$. At higher boosts, detector resolution effects will become more important, and with the higher pile-up expected at Run II of the LHC, pile-up mitigation will be crucial for future jet substructure studies. Future studies will be needed to determine which of the observables we have studied are most robust against pile-up and detector effects, and our analyses suggest particularly useful combinations of observables to consider in such studies. 

{\bf Have something snazzy to end the conclusion?}

%Ideas for general conclusions:
%\begin{itemize}
%\item It is clear from both the q/g tagging and W tagging studies that the correlation structure between the observables considered is complicated, being both \pt and R dependent.
%\end{itemize}

%mention about the pile-up
