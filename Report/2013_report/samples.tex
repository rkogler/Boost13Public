\subsection{Quark/gluon and $W$ tagging}

Samples were generated at $\sqrt{s} = 8\TeV$ for QCD dijets and $W^+W^-$
pairs decaying hadronically off a (pseudo) scalar resonance. The QCD events
were split into subsamples of $gg$ and $q\bar{q}$ events, allowing for tests of
discrimination of hadronic $W$ bosons, quarks, and gluons.

Individual quark and gluon samples were produced at leading order (LO)
using \textsc{MadGraph5}, while $W^+W^-$ samples were generated using
the \textsc{JHU Generator} to allow for separation of longitudinal and
transverse polarizations. Both were produced in exclusive $p_T$ bins
of 100 {\GeV} and generated using \textsc{CTEQ6L1} PDFs. The slicing parameter
was chosen to be the $p_T$ of any final state parton or $W$. At the
parton-level the \pt bins investigated were 300-400 GeV, 500-600 GeV
and 1.0-1.1 TeV. Since
no matching was performed, a cut on any parton was equivalent. These were
then showered through \textsc{Pythia8} (version 8.176) using the default tune 4C.

The showered events were clustered with \textsc{FastJet} 3.03 using
the anti-$k_t$ algorithm with jet radii of $R = 0.4,\, 0.8,\, 1.2$. In
both signal and background, an upper and lower cut on
the leading jet $\pt$ is applied after showering/clustering, to ensure
similar $\pt$ spectra for signal and background in each bin. The bins
in leading jet \pt that are investigated in the W-tagging and
q/g tagging studies are 300-450 GeV, 500-650 GeV, 1.0-1.2 TeV.

\subsection{Top tagging}
Samples were generated at $\sqrt{s}=14\TeV$. Standard Model dijet and top pair
samples were produced with \textsc{Sherpa} 2.0.0, with matrix elements of up
to two extra partons matched to the shower. The top samples included only
hadronic decays and  were generated in exclusive $\pt$ bins of width 100 \GeV,
taking as slicing parameter the maximum of the top/anti-top $\pt$. The QCD
samples were generated with a cut on the leading parton-level jet $\pt$, where
parton-level jets are clustered with the anti-$k_t$ algorithm and jet radii of
$R= 0.4,\,0.8,\,1.2$. The matching scale is selected to be $Q_{\rm cut}=40, 60, 80 \GeV$ for
the $p_{T\,\text{min}}=600, 1000$, and $1500 \GeV$ bins, respectively.
 
The analysis again relies on \textsc{FastJet} 3.0.3 for jet clustering and
calculation of jet substructure observables, and an upper and lower $\pt$ cut are applied
to each sample to ensure similar $\pt$ spectra in each bin. The bins in leading jet $\pt$
that are investigated for top tagging are 600-700 GeV, 1-1.1 TeV, and 1.5-1.6 TeV.