\noindent

Multivariate techniques are used to combine variables into an optimal
discriminant, and the extent to which the discrimination power
increases when this is done is used to indicate how much the
discriminatory information present in the variables overlaps. An
alternative strategy for studying correlations in discrimination power
that is not explored here  is ``truth
matching''~\cite{Larkoski:2014pca}.

In all cases the multivariate technique used to combine variables is a boosted decision tree (BDT) as implemented in the TMVA package~\cite{Hocker:2007ht}.
We use the BDT implementation including gradient boost.  
An example of the BDT settings are as follows: 
\begin{itemize}
\item NTrees=1000
\item BoostType=Grad
\item Shrinkage=0.1
\item UseBaggedGrad=F
\item nCuts=10000
\item MaxDepth=3
\item UseYesNoLeaf=F
\item nEventsMin=200
\end{itemize}
Exact parameter values are chosen to best reduce the effect of overtraining.
Additionally, the simulated data were split into training and testing samples and comparisons of the BDT output were compared to reduced the effect of overtraining as well.  
