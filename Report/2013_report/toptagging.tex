In this section, we study the identification of boosted top quarks at Run II of the LHC. Boosted top quarks result in large-radius jets with complex substructure, containing a $b$-subjet and a boosted $W$. The additional kinematic handles coming from the reconstruction of the $W$ mass and $b$-tagging allows a very high degree of discrimination of top quark jets from QCD backgrounds. 

We consider top quarks with moderate boost (600-1000 GeV), and perhaps most interestingly, at high boost ($\gtrsim1500$ GeV). Top tagging faces several challenges in the high-$\pt$ regime. For such high-$\pt$ jets, the $b$-tagging efficiencies are no longer reliably known. Also, the top jet can also accompanied by additional radiation with $\pt\sim m_t$, leading to combinatoric ambiguities of reconstructing the top and $W$, and the possibility that existing taggers or observables shape the background by looking for subjet combinations that reconstruct $m_t$/$m_W$. To study this, we examine the performance of both mass-reconstruction variables, as well as shape observables that probe the three-pronged nature of the top jet and the accompanying radiation pattern.

\subsection{Methodology}
We study a number of top-tagging strategies, in particular:
%
\begin{enumerate}
\item HEPTopTagger
\item Johns Hopkins Tagger (JH)
\item Trimming
\item Pruning
\end{enumerate}
%
The top taggers have criteria for reconstructing a top and $W$ candidate, while the grooming algorithms (trimming and pruning) do not incorporate a $W$-identification step. For a level playing field, we construct a $W$ candidate from the three leading subjets by taking the pair of subjets with the smallest invariant mass; in the case that only two subjets are reconstructed, we take the mass of the leading subjet. All of the above taggers and groomers incorporate a step to remove pile-up and other soft radiation.

We also consider the performance of jet shape observables. In particular, we consider the $N$-subjettiness ratios $\tau_{32}^{\beta=1}$ and $\tau_{21}^{\beta=1}$, energy correlation function ratios $C_3^{\beta=1}$ and $C_2^{\beta=1}$, and the Qjet mass volatility $\Gamma$. In addition to the jet shape performance, we combine the jet shapes with the mass-reconstruction methods listed above to determine the optimal combined performance.

For determining the performance of multiple variables, we combine the relevant tagger output observables and/or jet shapes into a boosted decision tree (BDT), which determines the optimal cut. Additionally, because each tagger has two inputs (list, or maybe refer back to Section 3), we scan over reasonable values of the inputs to determine the optimal value for each top tagging signal efficiency. This allows a direct comparison of the optimized version of each tagger. The input values scanned for the various algorithms are:
%
\begin{itemize}
\item {\bf HEPTopTagger:} $m\in[30,100]$ GeV, $\mu\in[0.5,1]$
\item {\bf JH Tagger:} $\delta_p\in[0.02,0.15]$, $\delta_R\in[0.07,0.2]$
\item {\bf Trimming:} $f_{\rm cut}\in[0.02,0.14]$, $R_{\rm trim}\in[0.1,0.5]$
\item {\bf Pruning:} $z_{\rm cut}\in[0.02,0.14]$, $R_{\rm cut}\in[0.1,0.6]$
\end{itemize}

\subsection{Single-observable performance}\label{sec:single_variable}
We start by investigating the behaviour of individual jet substructure observables. Because of the rich, three-pronged structure of the top decay, it is expected that combinations of masses and jet shapes will far outperform single observables in identifying boosted tops. However, a study of the top-tagging performance of single variables facilitates a direct comparison with the $W$ tagging results in Section \ref{sec:wtagging}, and also allows a straightforward examination of the performance of each observable for different $\pt$ and jet radius.

Fig.~\ref{fig:single_variable_ROC} shows the ROC curves for each of the top-tagging observables, with the bare jet mass also plotted for comparison. Unlike $W$ tagging, the jet shape observables perform more poorly than jet mass. \emph{(Check reasoning: this argument due to Andrew Larkoski)}. As an example illustrating why this is the case, consider $N$-subjettiness. The $W$ is two-pronged and the top is three-pronged; therefore, we expect $\tau_{21}$ and $\tau_{32}$ to be the best-performant $N$-subjettiness ratio, respectively. However, $\tau_{21}$ also contains an implicit cut on the denominator, $\tau_1$, which is strongly correlated with jet mass. Therefore, $\tau_{21}$ combines both mass and shape information to some extent. By contrast, and as is clear in Fig.\ref{fig:single_variable_ROC}(a), the best shape for top tagging is $\tau_{32}$, which contains no information on the mass. Therefore, it is  unsurprising that the  shapes most useful for top tagging are less sensitive to the jet mass, and under-perform relative to the corresponding observables for $W$ tagging.

\begin{figure*}
\begin{center}
\subfigure[Jet shapes]{\includegraphics[width=0.49\textwidth]{./Figures/TTagging/single_variable/pT.1TeV.R.0.8/Rocs_shape.pdf}}
\subfigure[Top mass]{\includegraphics[width=0.49\textwidth]{./Figures/TTagging/single_variable/pT.1TeV.R.0.8/Rocs_top_mass.pdf}}
\subfigure[$W$ mass]{\includegraphics[width=0.49\textwidth]{./Figures/TTagging/single_variable/pT.1TeV.R.0.8/Rocs_w_mass.pdf}}
\caption{Comparison of single-variable top-tagging performance in the \pt 1000-1100 GeV bin using the anti-\kT, R=0.8 algorithm.}
\label{fig:single_variable_ROC}
\end{center}
\end{figure*}

Of the two top tagging algorithms, the Johns Hopkins (JH) tagger out-performs the HEPTopTagger in its signal-to-background separation of both the top and $W$ candidate masses, with larger discrepancy at higher $\pt$ and larger jet radius. In Fig.~\ref{fig:topmass_histogram_HEP_JH}, we show the histograms for the top mass output from the JH and HEPTopTagger for different $\pt$ and $R$, optimized at a signal efficiency of 30\%. The likely reason for this behavior is that, in the HEPTopTagger algorithm, the jet is filtered to select the five hardest subjets, and then three subjets are chosen which reconstruct the top mass. This requirement tends to shape a peak in the QCD background around $m_t$ for the HEPTopTagger, while the JH tagger has no such requirement. It has been suggested by Anders \emph{et al.} \cite{Anders:2013oga} that performance in the HEPTopTagger may be improved by selecting the three subjets reconstructing the top only among those that pass the $W$ mass constraints, which somewhat reduces the shaping of the background. Note that both the JH tagger and the HEPTopTagger are superior at using the $W$ candidate inside of the top for signal discrimination; this is because the the pruning and trimming algorithms do not have inherent $W$-identification steps and are not optimized for this purpose.

\begin{figure*}
\begin{center}
\subfigure[Johns Hopkins Tagger, $R=0.4$]{\includegraphics[width=0.4\textwidth]{./Figures/TTagging/single_variable/pT.1.5TeV.R.0.4/h_JH_mt_opt_mt.pdf}}
\subfigure[HEPTopTagger, $R=0.4$]{\includegraphics[width=0.4\textwidth]{./Figures/TTagging/single_variable/pT.1.5TeV.R.0.4/h_HEP_mt_opt_mt.pdf}}
\subfigure[Johns Hopkins Tagger, $R=0.8$]{\includegraphics[width=0.4\textwidth]{./Figures/TTagging/single_variable/pT.1.5TeV.R.0.8/h_JH_mt_opt_mt.pdf}}
\subfigure[HEPTopTagger, $R=0.8$]{\includegraphics[width=0.4\textwidth]{./Figures/TTagging/single_variable/pT.1.5TeV.R.0.8/h_HEP_mt_opt_mt.pdf}}
\subfigure[Johns Hopkins Tagger, $R=1.2$]{\includegraphics[width=0.4\textwidth]{./Figures/TTagging/single_variable/pT.1.5TeV.R.1.2/h_JH_mt_opt_mt.pdf}}
\subfigure[HEPTopTagger, $R=1.2$]{\includegraphics[width=0.4\textwidth]{./Figures/TTagging/single_variable/pT.1.5TeV.R.1.2/h_HEP_mt_opt_mt.pdf}}
\caption{Comparison of top mass reconstruction with the JH and HEPTopTaggers at different $R$ using the anti-\kT algorithm, $p_{\rm T}=1.5-1.6$ TeV. Each histogram is shown for the working point optimized for best performance with $m_t$ at signal efficiency 0.3 and is normalized to the fraction of events passing the tagger.}
\label{fig:topmass_histogram_HEP_JH}
\end{center}
\end{figure*}

\begin{figure*}
\begin{center}
\subfigure[Johns Hopkins Tagger, $R=0.4$]{\includegraphics[width=0.4\textwidth]{./Figures/TTagging/single_variable/pT.1.5TeV.R.0.4/h_JH_mt_opt_all.pdf}}
\subfigure[HEPTopTagger, $R=0.4$]{\includegraphics[width=0.4\textwidth]{./Figures/TTagging/single_variable/pT.1.5TeV.R.0.4/h_HEP_mt_opt_all.pdf}}
\subfigure[Johns Hopkins Tagger, $R=0.8$]{\includegraphics[width=0.4\textwidth]{./Figures/TTagging/single_variable/pT.1.5TeV.R.0.8/h_JH_mt_opt_all.pdf}}
\subfigure[HEPTopTagger, $R=0.8$]{\includegraphics[width=0.4\textwidth]{./Figures/TTagging/single_variable/pT.1.5TeV.R.0.8/h_HEP_mt_opt_all.pdf}}
\subfigure[Johns Hopkins Tagger, $R=1.2$]{\includegraphics[width=0.4\textwidth]{./Figures/TTagging/single_variable/pT.1.5TeV.R.1.2/h_JH_mt_opt_all.pdf}}
\subfigure[HEPTopTagger, $R=1.2$]{\includegraphics[width=0.4\textwidth]{./Figures/TTagging/single_variable/pT.1.5TeV.R.1.2/h_HEP_mt_opt_all.pdf}}
\caption{Comparison of top mass reconstruction with the JH and HEPTopTaggers at different $R$ using the anti-\kT algorithm, $p_{\rm T}=1.5-1.6$ TeV. Each histogram is shown for the working point optimized for best performance of the tagger at signal efficiency 0.3 and is normalized to the fraction of events passing the tagger.}
\label{fig:topmass_histogram_optall_HEP_JH}
\end{center}
\end{figure*}




We also directly compare the performance of top mass and jet shape observables for different jet $\pt$ and radius. The input parameters of the taggers, groomers, and shape variables are separately optimized for each $\pt$ and radius:\\

\begin{figure*}
\begin{center}
\subfigure[$C_2^{(\beta=1)}$]{\includegraphics[width=0.48\textwidth]{./Figures/TTagging/single_variable/pT_compare/Rocs_C2b1_pTcompare.pdf}}
\subfigure[$C_3^{(\beta=1)}$]{\includegraphics[width=0.48\textwidth]{./Figures/TTagging/single_variable/pT_compare/Rocs_C3b1_pTcompare.pdf}}
\subfigure[$\tau_{21}^{(\beta=1)}$]{\includegraphics[width=0.48\textwidth]{./Figures/TTagging/single_variable/pT_compare/Rocs_tau21b1_pTcompare.pdf}}
\subfigure[$\tau_{32}^{(\beta=1)}$]{\includegraphics[width=0.48\textwidth]{./Figures/TTagging/single_variable/pT_compare/Rocs_tau32b1_pTcompare.pdf}}
\subfigure[Qjet mass volatility]{\includegraphics[width=0.48\textwidth]{./Figures/TTagging/single_variable/pT_compare/Rocs_Qjet_pTcompare.pdf}}
\caption{Comparison of individual jet shape performance at different \pt using the anti-\kT R=0.8 algorithm.}
\label{fig:ptcomparison_singleshape_top}
\end{center}
\end{figure*}

\begin{figure*}
\begin{center}
\subfigure[HEPTopTagger $m_t$]{\includegraphics[width=0.48\textwidth]{./Figures/TTagging/single_variable/pT_compare/Rocs_HEP_mt_pTcompare.pdf}}
\subfigure[Johns Hopkins Tagger $m_t$]{\includegraphics[width=0.48\textwidth]{./Figures/TTagging/single_variable/pT_compare/Rocs_JH_mt_pTcompare.pdf}}
\subfigure[Pruning $m_t$]{\includegraphics[width=0.48\textwidth]{./Figures/TTagging/single_variable/pT_compare/Rocs_prune_pTcompare.pdf}}
\subfigure[Trimming $m_t$]{\includegraphics[width=0.48\textwidth]{./Figures/TTagging/single_variable/pT_compare/Rocs_trim_pTcompare.pdf}}
\caption{Comparison of top mass performance of different taggers at different \pt using the anti-\kT R=0.8 algorithm.}
\label{fig:ptcomparison_singletopmass_top}
\end{center}
\end{figure*}

\noindent{\bf $\pt$ comparison:} We compare various top tagging observables for jets in different $\pt$ bins and $R=0.8$ in Figs.~\ref{fig:ptcomparison_singleshape_top} and \ref{fig:ptcomparison_singletopmass_top}. The tagging performance of jet shapes do not change substantially with $\pt$. $\tau_{32}^{(\beta=1)}$ and the Qjet volatility $\Gamma$ have the most variation and tend to degrade with higher $\pt$. This makes sense, as higher-$\pt$ QCD jets have more, harder emissions within the jet, giving rise to substructure that fakes the signal. By contrast, most of the top mass observables have superior performance at higher $\pt$ due to the radiation from the top quark becoming more collimated. The notable exception is the HEPTopTagger, which degrades at higher $\pt$, likely in part due to the background-shaping effects discussed earlier.\\

\begin{figure*}
\begin{center}
\subfigure[$C_2^{(\beta=1)}$]{\includegraphics[width=0.48\textwidth]{./Figures/TTagging/single_variable/R_compare/Rocs_C2b1_Rcompare.pdf}}
\subfigure[$C_3^{(\beta=1)}$]{\includegraphics[width=0.48\textwidth]{./Figures/TTagging/single_variable/R_compare/Rocs_C3b1_Rcompare.pdf}}
\subfigure[$\tau_{21}^{(\beta=1)}$]{\includegraphics[width=0.48\textwidth]{./Figures/TTagging/single_variable/R_compare/Rocs_tau21b1_Rcompare.pdf}}
\subfigure[$\tau_{32}^{(\beta=1)}$]{\includegraphics[width=0.48\textwidth]{./Figures/TTagging/single_variable/R_compare/Rocs_tau32b1_Rcompare.pdf}}
\subfigure[Qjet mass volatility]{\includegraphics[width=0.48\textwidth]{./Figures/TTagging/single_variable/R_compare/Rocs_Qjet_Rcompare.pdf}}
\caption{Comparison of individual jet shape performance at different $R$ in the $\pt=1500-1600$ GeV bin.}
\label{fig:Rcomparison_singleshape_top}
\end{center}
\end{figure*}

\begin{figure*}
\begin{center}
\subfigure[HEPTopTagger $m_t$]{\includegraphics[width=0.48\textwidth]{./Figures/TTagging/single_variable/R_compare/Rocs_HEP_mt_Rcompare.pdf}}
\subfigure[Johns Hopkins Tagger $m_t$]{\includegraphics[width=0.48\textwidth]{./Figures/TTagging/single_variable/R_compare/Rocs_JH_mt_Rcompare.pdf}}
\subfigure[Pruning $m_t$]{\includegraphics[width=0.48\textwidth]{./Figures/TTagging/single_variable/R_compare/Rocs_prune_Rcompare.pdf}}
\subfigure[Trimming $m_t$]{\includegraphics[width=0.48\textwidth]{./Figures/TTagging/single_variable/R_compare/Rocs_trim_Rcompare.pdf}}
\caption{Comparison of top mass performance of different taggers at different $R$ in the $\pt=1500-1600$ GeV bin.}
\label{fig:Rcomparison_singletopmass_top}
\end{center}
\end{figure*}

\noindent{\bf $R$ comparison:} We compare various top tagging observables for jets of different $R$ and $\pt=1.5-1.6$ TeV in Figs.~\ref{fig:Rcomparison_singleshape_top}-\ref{fig:Rcomparison_singletopmass_top}. Most of the top-tagging parameters perform best for smaller radius; this is because, at such high $\pt$, most of the radiation from the top quark is confined within $R=0.4$, and having a larger jet radius makes the observable more susceptible to contamination from the underlying event and other uncorrelated radiation. The main exception is $C_3^{(\beta=1)}$, which performs optimally at $R=0.8$. \emph{why?}




\subsection{Performance of multivariable combinations}
We now consider various combinations of the observables from Section \ref{sec:single_variable}. In particular, we consider the performance of individual taggers such as the JH tagger and HEPTopTagger, which output information about the $t$ and $W$ candidate masses and the helicity angle; groomers, such as trimming and pruning, which remove soft, uncorrelated radiation from the top candidate to improve mass reconstruction, and to which we have added a $W$ reconstruction step; and the combination of the above taggers/groomers with shape variables such as $N$-subjettiness ratios and energy correlation ratios. For all observables with tuneable input parameters, we scan and optimize over realistic values of such parameters.

\emph{Link to discussion of BDT methods}

Fig.~\ref{fig:ptcomparison_top} shows our main results for the multivariable combinations; in all cases, we also show the ungroomed jet mass as a baseline comparison. In Fig.~\ref{fig:ptcomparison_top}(a), we directly compare the performance of the HEPTopTagger, the JH tagger, trimming, and grooming. Generally, we find that pruning, which does not naturally incorporate subjets into the algorithm, does not perform as well as the others. Interestingly, trimming, which does include a subjet-identification step, performs comparably to the HEPTopTagger over much of the range, possibly due to the background-shaping observed in Section \ref{sec:single_variable}. By contrast, the JH tagger outperforms the other algorithms.

To determine whether there is complementary information in the mass outputs from different top taggers, we also consider a multivariable combination of all of the JH and HEPTopTagger outputs. The maximum efficiency of the combined JH and HEPTopTaggers is limited, as some fraction of signal events inevitably fails either one or other of the taggers. We do see a 20-50\% improvement in performance when combining all outputs, which suggests that the different algorithms used to identify the $t$ and $W$ for different taggers contains complementary information.

 In Fig.~\ref{fig:ptcomparison_top}(b)-(d), we present the results for multivariable combinations of top tagger outputs with and without shape variables. We see that, for both the HEPTopTagger and the JH tagger, the shape observables contain additional information uncorrelated with the masses and helicity angle, and give on average 2-3 improvement in signal discrimination. We see that, when combined with the tagger outputs, both the energy correlation functions $C_2+C_3$ and the $N$-subjettiness ratios $\tau_{21}+\tau_{32}$ give comparable performance, while the Qjet mass volatility is slightly worse; this is unsurprising, as Qjets accesses shape information in a more indirect way from other shape observables. \emph{OK?} Combining all shape observables with a single top tagger provides  even more  enhancement in discrimination power.

We directly compare the performance of the JH and HEPTopTaggers in Fig.~\ref{fig:ptcomparison_top}(d). Combining the taggers with shape information nearly erases the difference between the tagging methods observed in Fig.~\ref{fig:ptcomparison_top}(a); this indicates that combining the shape information with the HEPTopTagger identifies the differences between signal and background missed by the tagger alone. This also suggests that further improvement to discriminating power may be minimal, as various multivariable combinations are converging to within a factor of 20\% or so.

In Fig.~\ref{fig:ptcomparison_top}(e)-(g), we present the results for multivariable combinations of groomer outputs with and without shape variables. As with the tagging algorithms, combinations of groomers with shape observables improves their discriminating power; combinations with $\tau_{32}+\tau_{21}$ perform comparably to those with $C_3+C_2$, and both of these are superior to combinations with the mass volatility, $\Gamma$. Substantial improvement is further possible by combining the groomers with all shape observables. Once again, in \ref{fig:ptcomparison_top}(g), we find that the differences between pruning and trimming are erased when combined with shape information.


\begin{figure*}
\begin{center}
\includegraphics[width=0.7\textwidth]{./Figures/TTagging/multi_variable/pT.1TeV.R.0.8/Rocs_tagger_groom.pdf}
\caption{Comparison of BDT combinations of each tagger output in the \pt 1000-1100 GeV bin using the anti-\kT R=0.8 algorithm.}
\label{fig:pt1000_taggers_AKt_R08}
\end{center}
\end{figure*}


\begin{figure*}
\begin{center}
\subfigure[Tagger vs.~Groomer]{\includegraphics[width=0.48\textwidth]{./Figures/TTagging/multi_variable/pT.1TeV.R.0.8/Rocs_tagger_groom.pdf}}
\subfigure[HEPTopTagger]{\includegraphics[width=0.48\textwidth]{./Figures/TTagging/multi_variable/pT.1TeV.R.0.8/Rocs_HEP.pdf}}
\subfigure[Johns Hopkins Tagger]{\includegraphics[width=0.48\textwidth]{./Figures/TTagging/multi_variable/pT.1TeV.R.0.8/Rocs_JH.pdf}}
\subfigure[HEP+JH comparison]{\includegraphics[width=0.48\textwidth]{./Figures/TTagging/multi_variable/pT.1TeV.R.0.8/Rocs_tagger_shape.pdf}}
\subfigure[Pruning]{\includegraphics[width=0.48\textwidth]{./Figures/TTagging/multi_variable/pT.1TeV.R.0.8/Rocs_prune.pdf}}
\subfigure[Trimming]{\includegraphics[width=0.48\textwidth]{./Figures/TTagging/multi_variable/pT.1TeV.R.0.8/Rocs_trim.pdf}}
\subfigure[Grooming comparison]{\includegraphics[width=0.48\textwidth]{./Figures/TTagging/multi_variable/pT.1TeV.R.0.8/Rocs_groom_shape.pdf}}
\subfigure[Comparison of Tagger+Shape]{\includegraphics[width=0.48\textwidth]{./Figures/TTagging/multi_variable/pT.1TeV.R.0.8/Rocs_optimum.pdf}}
\caption{The BDT combinations in the \pt 1000-1100 GeV bin using the anti-\kT R=0.8 algorithm.}
\label{fig:pt1000_taggers_AKt_R08}
\end{center}
\end{figure*}

\clearpage


\subsubsection{$\pt$ comparison}

\begin{figure*}
\begin{center}
\subfigure[HEPTopTagger]{\includegraphics[width=0.48\textwidth]{./Figures/TTagging/multi_variable/pT_compare/Rocs_HEP_pTcompare.pdf}}
\subfigure[Johns Hopkins Tagger]{\includegraphics[width=0.48\textwidth]{./Figures/TTagging/multi_variable/pT_compare/Rocs_JH_pTcompare.pdf}}
\subfigure[JH + HEP taggers]{\includegraphics[width=0.48\textwidth]{./Figures/TTagging/multi_variable/pT_compare/Rocs_HEP_JH_pTcompare.pdf}}
\subfigure[Trimming]{\includegraphics[width=0.48\textwidth]{./Figures/TTagging/multi_variable/pT_compare/Rocs_trim_mt_mw_pTcompare.pdf}}
\subfigure[Pruning]{\includegraphics[width=0.48\textwidth]{./Figures/TTagging/multi_variable/pT_compare/Rocs_prune_mt_mw_pTcompare.pdf}}
\caption{Comparison of BDT combination of tagger performance at different \pt using the anti-\kT R=0.8 algorithm.}
\label{fig:ptcomparison_top}
\end{center}
\end{figure*}

\begin{figure*}
\begin{center}
\subfigure[JH+$C_2^{(\beta=1)}$+$C_3^{(\beta=1)}$]{\includegraphics[width=0.48\textwidth]{./Figures/TTagging/multi_variable/pT_compare/Rocs_JH_C_pTcompare.pdf}}
\subfigure[JH+$\tau_{21}^{(\beta=1)}$+$\tau_{32}^{(\beta=1)}$]{\includegraphics[width=0.48\textwidth]{./Figures/TTagging/multi_variable/pT_compare/Rocs_JH_tau_pTcompare.pdf}}
\subfigure[JH + Qjet mass volatility]{\includegraphics[width=0.48\textwidth]{./Figures/TTagging/multi_variable/pT_compare/Rocs_JH_Qjet_pTcompare.pdf}}
\subfigure[JH + all]{\includegraphics[width=0.48\textwidth]{./Figures/TTagging/multi_variable/pT_compare/Rocs_JH_shape_pTcompare.pdf}}
\caption{Comparison of BDT combination of JH tagger + shape at different \pt using the anti-\kT R=0.8 algorithm.}
\label{fig:ptcomparison_JH_shape}
\end{center}
\end{figure*}

\begin{figure*}
\begin{center}
\subfigure[HEP+$C_2^{(\beta=1)}$+$C_3^{(\beta=1)}$]{\includegraphics[width=0.48\textwidth]{./Figures/TTagging/multi_variable/pT_compare/Rocs_HEP_C_pTcompare.pdf}}
\subfigure[HEP+$\tau_{21}^{(\beta=1)}$+$\tau_{32}^{(\beta=1)}$]{\includegraphics[width=0.48\textwidth]{./Figures/TTagging/multi_variable/pT_compare/Rocs_HEP_tau_pTcompare.pdf}}
\subfigure[HEP + Qjet mass volatility]{\includegraphics[width=0.48\textwidth]{./Figures/TTagging/multi_variable/pT_compare/Rocs_HEP_Qjet_pTcompare.pdf}}
\subfigure[HEP + all]{\includegraphics[width=0.48\textwidth]{./Figures/TTagging/multi_variable/pT_compare/Rocs_HEP_shape_pTcompare.pdf}}
\caption{Comparison of BDT combination of HEP tagger + shape at different \pt using the anti-\kT R=0.8 algorithm.}
\label{fig:ptcomparison_HEP_shape}
\end{center}
\end{figure*}

\begin{figure*}
\begin{center}
\subfigure[trim $m_t$+$m_W$+$C_2^{(\beta=1)}$+$C_3^{(\beta=1)}$]{\includegraphics[width=0.48\textwidth]{./Figures/TTagging/multi_variable/pT_compare/Rocs_trim_mt_mw_C_pTcompare.pdf}}
\subfigure[trim $m_t$+$m_W$+$\tau_{21}^{(\beta=1)}$+$\tau_{32}^{(\beta=1)}$]{\includegraphics[width=0.48\textwidth]{./Figures/TTagging/multi_variable/pT_compare/Rocs_trim_mt_mw_tau_pTcompare.pdf}}
\subfigure[trim $m_t$+$m_W$ + Qjet mass volatility]{\includegraphics[width=0.48\textwidth]{./Figures/TTagging/multi_variable/pT_compare/Rocs_trim_mt_mw_Qjet_pTcompare.pdf}}
\subfigure[trim $m_t$+$m_W$ + all]{\includegraphics[width=0.48\textwidth]{./Figures/TTagging/multi_variable/pT_compare/Rocs_trim_mt_mw_shape_pTcompare.pdf}}
\caption{Comparison of BDT combination of trimming + shape at different \pt using the anti-\kT R=0.8 algorithm.}
\label{fig:ptcomparison_trim_shape}
\end{center}
\end{figure*}

\begin{figure*}
\begin{center}
\subfigure[prune $m_t$+$m_W$+$C_2^{(\beta=1)}$+$C_3^{(\beta=1)}$]{\includegraphics[width=0.48\textwidth]{./Figures/TTagging/multi_variable/pT_compare/Rocs_prune_mt_mw_C_pTcompare.pdf}}
\subfigure[prune $m_t$+$m_W$+$\tau_{21}^{(\beta=1)}$+$\tau_{32}^{(\beta=1)}$]{\includegraphics[width=0.48\textwidth]{./Figures/TTagging/multi_variable/pT_compare/Rocs_prune_mt_mw_tau_pTcompare.pdf}}
\subfigure[prune $m_t$+$m_W$ + Qjet mass volatility]{\includegraphics[width=0.48\textwidth]{./Figures/TTagging/multi_variable/pT_compare/Rocs_prune_mt_mw_Qjet_pTcompare.pdf}}
\subfigure[prune $m_t$+$m_W$ + all]{\includegraphics[width=0.48\textwidth]{./Figures/TTagging/multi_variable/pT_compare/Rocs_prune_mt_mw_shape_pTcompare.pdf}}
\caption{Comparison of BDT combination of pruning + shape at different \pt using the anti-\kT R=0.8 algorithm.}
\label{fig:ptcomparison_prune_shape}
\end{center}
\end{figure*}

\clearpage
\subsubsection{$R$ comparison}

\begin{figure*}
\begin{center}
\subfigure[HEPTopTagger]{\includegraphics[width=0.48\textwidth]{./Figures/TTagging/multi_variable/R_compare/Rocs_HEP_Rcompare.pdf}}
\subfigure[Johns Hopkins Tagger]{\includegraphics[width=0.48\textwidth]{./Figures/TTagging/multi_variable/R_compare/Rocs_JH_Rcompare.pdf}}
\subfigure[JH + HEP taggers]{\includegraphics[width=0.48\textwidth]{./Figures/TTagging/multi_variable/R_compare/Rocs_HEP_JH_Rcompare.pdf}}
\subfigure[Trimming]{\includegraphics[width=0.48\textwidth]{./Figures/TTagging/multi_variable/R_compare/Rocs_trim_mt_mw_Rcompare.pdf}}
\subfigure[Pruning]{\includegraphics[width=0.48\textwidth]{./Figures/TTagging/multi_variable/R_compare/Rocs_prune_mt_mw_Rcompare.pdf}}
\caption{Comparison of tagger and jet shape performance at different radius at \pt = 1.5-1.6 TeV.}
\label{fig:Rcomparison_top}
\end{center}
\end{figure*}

\begin{figure*}
\begin{center}
\subfigure[JH+$C_2^{(\beta=1)}$+$C_3^{(\beta=1)}$]{\includegraphics[width=0.48\textwidth]{./Figures/TTagging/multi_variable/R_compare/Rocs_JH_C_Rcompare.pdf}}
\subfigure[JH+$\tau_{21}^{(\beta=1)}$+$\tau_{32}^{(\beta=1)}$]{\includegraphics[width=0.48\textwidth]{./Figures/TTagging/multi_variable/R_compare/Rocs_JH_tau_Rcompare.pdf}}
\subfigure[JH + Qjet mass volatility]{\includegraphics[width=0.48\textwidth]{./Figures/TTagging/multi_variable/R_compare/Rocs_JH_Qjet_Rcompare.pdf}}
\subfigure[JH + all]{\includegraphics[width=0.48\textwidth]{./Figures/TTagging/multi_variable/R_compare/Rocs_JH_shape_Rcompare.pdf}}
\caption{Comparison of BDT combination of JH tagger + shape at different radius at \pt = 1.5-1.6 TeV.}
\label{fig:Rcomparison_JH_shape}
\end{center}
\end{figure*}

\begin{figure*}
\begin{center}
\subfigure[HEP+$C_2^{(\beta=1)}$+$C_3^{(\beta=1)}$]{\includegraphics[width=0.48\textwidth]{./Figures/TTagging/multi_variable/R_compare/Rocs_HEP_C_Rcompare.pdf}}
\subfigure[HEP+$\tau_{21}^{(\beta=1)}$+$\tau_{32}^{(\beta=1)}$]{\includegraphics[width=0.48\textwidth]{./Figures/TTagging/multi_variable/R_compare/Rocs_HEP_tau_Rcompare.pdf}}
\subfigure[HEP + Qjet mass volatility]{\includegraphics[width=0.48\textwidth]{./Figures/TTagging/multi_variable/R_compare/Rocs_HEP_Qjet_Rcompare.pdf}}
\subfigure[HEP + all]{\includegraphics[width=0.48\textwidth]{./Figures/TTagging/multi_variable/R_compare/Rocs_HEP_shape_Rcompare.pdf}}
\caption{Comparison of BDT combination of HEP tagger + shape at different radius at \pt = 1.5-1.6 TeV.}
\label{fig:Rcomparison_HEP_shape}
\end{center}
\end{figure*}

\begin{figure*}
\begin{center}
\subfigure[trim $m_t$+$m_W$+$C_2^{(\beta=1)}$+$C_3^{(\beta=1)}$]{\includegraphics[width=0.48\textwidth]{./Figures/TTagging/multi_variable/R_compare/Rocs_trim_mt_mw_C_Rcompare.pdf}}
\subfigure[trim $m_t$+$m_W$+$\tau_{21}^{(\beta=1)}$+$\tau_{32}^{(\beta=1)}$]{\includegraphics[width=0.48\textwidth]{./Figures/TTagging/multi_variable/R_compare/Rocs_trim_mt_mw_tau_Rcompare.pdf}}
\subfigure[trim $m_t$+$m_W$ + Qjet mass volatility]{\includegraphics[width=0.48\textwidth]{./Figures/TTagging/multi_variable/R_compare/Rocs_trim_mt_mw_Qjet_Rcompare.pdf}}
\subfigure[trim $m_t$+$m_W$ + all]{\includegraphics[width=0.48\textwidth]{./Figures/TTagging/multi_variable/R_compare/Rocs_trim_mt_mw_shape_Rcompare.pdf}}
\caption{Comparison of BDT combination of trimming + shape at different radius at \pt = 1.5-1.6 TeV.}
\label{fig:Rcomparison_trim_shape}
\end{center}
\end{figure*}

\begin{figure*}
\begin{center}
\subfigure[prune $m_t$+$m_W$+$C_2^{(\beta=1)}$+$C_3^{(\beta=1)}$]{\includegraphics[width=0.48\textwidth]{./Figures/TTagging/multi_variable/R_compare/Rocs_prune_mt_mw_C_Rcompare.pdf}}
\subfigure[prune $m_t$+$m_W$+$\tau_{21}^{(\beta=1)}$+$\tau_{32}^{(\beta=1)}$]{\includegraphics[width=0.48\textwidth]{./Figures/TTagging/multi_variable/R_compare/Rocs_prune_mt_mw_tau_Rcompare.pdf}}
\subfigure[prune $m_t$+$m_W$ + Qjet mass volatility]{\includegraphics[width=0.48\textwidth]{./Figures/TTagging/multi_variable/R_compare/Rocs_prune_mt_mw_Qjet_Rcompare.pdf}}
\subfigure[prune $m_t$+$m_W$ + all]{\includegraphics[width=0.48\textwidth]{./Figures/TTagging/multi_variable/R_compare/Rocs_prune_mt_mw_shape_Rcompare.pdf}}
\caption{Comparison of BDT combination of pruning + shape at different radius at \pt = 1.5-1.6 TeV.}
\label{fig:Rcomparison_prune_shape}
\end{center}
\end{figure*}



\clearpage
\subsection{Performance at Sub-Optimal Working Points}

\subsubsection{$\pt$ dependence (single variable)}

\begin{figure*}
\begin{center}
\subfigure[$C_2^{(\beta=1)}$]{\includegraphics[width=0.48\textwidth]{./Figures/TTagging/single_variable/pT_compare/Rocs_C2b1_pTcompare_optOnce.pdf}}
\subfigure[$C_3^{(\beta=1)}$]{\includegraphics[width=0.48\textwidth]{./Figures/TTagging/single_variable/pT_compare/Rocs_C3b1_pTcompare_optOnce.pdf}}
\subfigure[$\tau_{21}^{(\beta=1)}$]{\includegraphics[width=0.48\textwidth]{./Figures/TTagging/single_variable/pT_compare/Rocs_tau21b1_pTcompare_optOnce.pdf}}
\subfigure[$\tau_{32}^{(\beta=1)}$]{\includegraphics[width=0.48\textwidth]{./Figures/TTagging/single_variable/pT_compare/Rocs_tau32b1_pTcompare_optOnce.pdf}}
\subfigure[Qjet mass volatility]{\includegraphics[width=0.48\textwidth]{./Figures/TTagging/single_variable/pT_compare/Rocs_Qjet_pTcompare_optOnce.pdf}}
\caption{Comparison of individual jet shape performance at different \pt using the anti-\kT R=0.8 algorithm; the tagger inputs are set to the optimum value for $\pt=1.5-1.6$ TeV.}
\label{fig:ptcomparison_singleshape_top_optOnce}
\end{center}
\end{figure*}

\begin{figure*}
\begin{center}
\subfigure[HEPTopTagger $m_t$]{\includegraphics[width=0.48\textwidth]{./Figures/TTagging/single_variable/pT_compare/Rocs_HEP_mt_pTcompare_optOnce.pdf}}
\subfigure[Johns Hopkins Tagger $m_t$]{\includegraphics[width=0.48\textwidth]{./Figures/TTagging/single_variable/pT_compare/Rocs_JH_mt_pTcompare_optOnce.pdf}}
\subfigure[Pruning $m_t$]{\includegraphics[width=0.48\textwidth]{./Figures/TTagging/single_variable/pT_compare/Rocs_prune_pTcompare_optOnce.pdf}}
\subfigure[Trimming $m_t$]{\includegraphics[width=0.48\textwidth]{./Figures/TTagging/single_variable/pT_compare/Rocs_trim_pTcompare_optOnce.pdf}}
\caption{Comparison of top mass performance of different taggers at different \pt using the anti-\kT R=0.8 algorithm; the tagger inputs are set to the optimum value for $\pt=1.5-1.6$ TeV.}
\label{fig:ptcomparison_singletopmass_top_optOnce}
\end{center}
\end{figure*}

\begin{figure*}
\begin{center}
\subfigure[HEPTopTagger $m_W$]{\includegraphics[width=0.48\textwidth]{./Figures/TTagging/single_variable/pT_compare/Rocs_HEP_mw_pTcompare_optOnce.pdf}}
\subfigure[Johns Hopkins Tagger $m_W$]{\includegraphics[width=0.48\textwidth]{./Figures/TTagging/single_variable/pT_compare/Rocs_JH_mw_pTcompare_optOnce.pdf}}
\subfigure[Pruning $m_W$]{\includegraphics[width=0.48\textwidth]{./Figures/TTagging/single_variable/pT_compare/Rocs_prune_mw_pTcompare_optOnce.pdf}}
\subfigure[Trimming $m_W$]{\includegraphics[width=0.48\textwidth]{./Figures/TTagging/single_variable/pT_compare/Rocs_trim_mw_pTcompare_optOnce.pdf}}
\caption{Comparison of $W$ mass performance of different taggers at different \pt using the anti-\kT R=0.8 algorithm; the tagger inputs are set to the optimum value for $\pt=1.5-1.6$ TeV.}
\label{fig:ptcomparison_singlewmass_top_optOnce}
\end{center}
\end{figure*}

\clearpage
\subsubsection{$R$ dependence (single variable)}

\begin{figure*}
\begin{center}
\subfigure[$C_2^{(\beta=1)}$]{\includegraphics[width=0.48\textwidth]{./Figures/TTagging/single_variable/R_compare/Rocs_C2b1_Rcompare_optOnce.pdf}}
\subfigure[$C_3^{(\beta=1)}$]{\includegraphics[width=0.48\textwidth]{./Figures/TTagging/single_variable/R_compare/Rocs_C3b1_Rcompare_optOnce.pdf}}
\subfigure[$\tau_{21}^{(\beta=1)}$]{\includegraphics[width=0.48\textwidth]{./Figures/TTagging/single_variable/R_compare/Rocs_tau21b1_Rcompare_optOnce.pdf}}
\subfigure[$\tau_{32}^{(\beta=1)}$]{\includegraphics[width=0.48\textwidth]{./Figures/TTagging/single_variable/R_compare/Rocs_tau32b1_Rcompare_optOnce.pdf}}
\subfigure[Qjet mass volatility]{\includegraphics[width=0.48\textwidth]{./Figures/TTagging/single_variable/R_compare/Rocs_Qjet_Rcompare_optOnce.pdf}}
\caption{Comparison of individual jet shape performance at different $R$ in the $\pt=1500-1600$ GeV bin; the tagger inputs are set to the optimum value for $R=1.2$ TeV.}
\label{fig:Rcomparison_singleshape_top_optOnce}
\end{center}
\end{figure*}

\begin{figure*}
\begin{center}
\subfigure[HEPTopTagger $m_t$]{\includegraphics[width=0.48\textwidth]{./Figures/TTagging/single_variable/R_compare/Rocs_HEP_mt_Rcompare_optOnce.pdf}}
\subfigure[Johns Hopkins Tagger $m_t$]{\includegraphics[width=0.48\textwidth]{./Figures/TTagging/single_variable/R_compare/Rocs_JH_mt_Rcompare_optOnce.pdf}}
\subfigure[Pruning $m_t$]{\includegraphics[width=0.48\textwidth]{./Figures/TTagging/single_variable/R_compare/Rocs_prune_Rcompare_optOnce.pdf}}
\subfigure[Trimming $m_t$]{\includegraphics[width=0.48\textwidth]{./Figures/TTagging/single_variable/R_compare/Rocs_trim_Rcompare_optOnce.pdf}}
\caption{Comparison of top mass performance of different taggers at different $R$ in the $\pt=1500-1600$ GeV bin; the tagger inputs are set to the optimum value for $R=1.2$ TeV.}
\label{fig:Rcomparison_singletopmass_top_optOnce}
\end{center}
\end{figure*}

\begin{figure*}
\begin{center}
\subfigure[HEPTopTagger $m_W$]{\includegraphics[width=0.48\textwidth]{./Figures/TTagging/single_variable/R_compare/Rocs_HEP_mw_Rcompare_optOnce.pdf}}
\subfigure[Johns Hopkins Tagger $m_W$]{\includegraphics[width=0.48\textwidth]{./Figures/TTagging/single_variable/R_compare/Rocs_JH_mw_Rcompare_optOnce.pdf}}
\subfigure[Pruning $m_W$]{\includegraphics[width=0.48\textwidth]{./Figures/TTagging/single_variable/R_compare/Rocs_prune_mw_Rcompare_optOnce.pdf}}
\subfigure[Trimming $m_W$]{\includegraphics[width=0.48\textwidth]{./Figures/TTagging/single_variable/R_compare/Rocs_trim_mw_Rcompare_optOnce.pdf}}
\caption{Comparison of $W$ mass performance of different taggers at different $R$ in the $\pt=1500-1600$ GeV bin; the tagger inputs are set to the optimum value for $R=1.2$ TeV.}
\label{fig:Rcomparison_singlewmass_top_optOnce}
\end{center}
\end{figure*}

\clearpage

\subsubsection{$\pt$ dependence}

\begin{figure*}
\begin{center}
\subfigure[HEPTopTagger]{\includegraphics[width=0.48\textwidth]{./Figures/TTagging/multi_variable/pT_compare/Rocs_HEP_pTcompare_optOnce.pdf}}
\subfigure[Johns Hopkins Tagger]{\includegraphics[width=0.48\textwidth]{./Figures/TTagging/multi_variable/pT_compare/Rocs_JH_pTcompare_optOnce.pdf}}
\subfigure[JH + HEP taggers]{\includegraphics[width=0.48\textwidth]{./Figures/TTagging/multi_variable/pT_compare/Rocs_HEP_JH_pTcompare_optOnce.pdf}}
\subfigure[Trimming]{\includegraphics[width=0.48\textwidth]{./Figures/TTagging/multi_variable/pT_compare/Rocs_trim_mt_mw_pTcompare_optOnce.pdf}}
\subfigure[Pruning]{\includegraphics[width=0.48\textwidth]{./Figures/TTagging/multi_variable/pT_compare/Rocs_prune_mt_mw_pTcompare_optOnce.pdf}}
\caption{Comparison of BDT combination of tagger performance at different \pt using the anti-\kT R=0.8 algorithm; the tagger inputs are set to the optimum value for $\pt=1.5-1.6$ TeV.}
\label{fig:ptcomparison_top_optOnce}
\end{center}
\end{figure*}

\begin{figure*}
\begin{center}
\subfigure[JH+$C_2^{(\beta=1)}$+$C_3^{(\beta=1)}$]{\includegraphics[width=0.48\textwidth]{./Figures/TTagging/multi_variable/pT_compare/Rocs_JH_C_pTcompare_optOnce.pdf}}
\subfigure[JH+$\tau_{21}^{(\beta=1)}$+$\tau_{32}^{(\beta=1)}$]{\includegraphics[width=0.48\textwidth]{./Figures/TTagging/multi_variable/pT_compare/Rocs_JH_tau_pTcompare_optOnce.pdf}}
\subfigure[JH + Qjet mass volatility]{\includegraphics[width=0.48\textwidth]{./Figures/TTagging/multi_variable/pT_compare/Rocs_JH_Qjet_pTcompare_optOnce.pdf}}
\subfigure[JH + all]{\includegraphics[width=0.48\textwidth]{./Figures/TTagging/multi_variable/pT_compare/Rocs_JH_shape_pTcompare_optOnce.pdf}}
\caption{Comparison of BDT combination of JH tagger + shape at different \pt using the anti-\kT R=0.8 algorithm; the tagger inputs are set to the optimum value for $\pt=1.5-1.6$ TeV.}
\label{fig:ptcomparison_JH_shape_optOnce}
\end{center}
\end{figure*}

\begin{figure*}
\begin{center}
\subfigure[HEP+$C_2^{(\beta=1)}$+$C_3^{(\beta=1)}$]{\includegraphics[width=0.48\textwidth]{./Figures/TTagging/multi_variable/pT_compare/Rocs_HEP_C_pTcompare_optOnce.pdf}}
\subfigure[HEP+$\tau_{21}^{(\beta=1)}$+$\tau_{32}^{(\beta=1)}$]{\includegraphics[width=0.48\textwidth]{./Figures/TTagging/multi_variable/pT_compare/Rocs_HEP_tau_pTcompare_optOnce.pdf}}
\subfigure[HEP + Qjet mass volatility]{\includegraphics[width=0.48\textwidth]{./Figures/TTagging/multi_variable/pT_compare/Rocs_HEP_Qjet_pTcompare_optOnce.pdf}}
\subfigure[HEP + all]{\includegraphics[width=0.48\textwidth]{./Figures/TTagging/multi_variable/pT_compare/Rocs_HEP_shape_pTcompare_optOnce.pdf}}
\caption{Comparison of BDT combination of HEP tagger + shape at different \pt using the anti-\kT R=0.8 algorithm; the tagger inputs are set to the optimum value for $\pt=1.5-1.6$ TeV.}
\label{fig:ptcomparison_HEP_shape_optOnce}
\end{center}
\end{figure*}

\begin{figure*}
\begin{center}
\subfigure[trim $m_t$+$m_W$+$C_2^{(\beta=1)}$+$C_3^{(\beta=1)}$]{\includegraphics[width=0.48\textwidth]{./Figures/TTagging/multi_variable/pT_compare/Rocs_trim_mt_mw_C_pTcompare_optOnce.pdf}}
\subfigure[trim $m_t$+$m_W$+$\tau_{21}^{(\beta=1)}$+$\tau_{32}^{(\beta=1)}$]{\includegraphics[width=0.48\textwidth]{./Figures/TTagging/multi_variable/pT_compare/Rocs_trim_mt_mw_tau_pTcompare_optOnce.pdf}}
\subfigure[trim $m_t$+$m_W$ + Qjet mass volatility]{\includegraphics[width=0.48\textwidth]{./Figures/TTagging/multi_variable/pT_compare/Rocs_trim_mt_mw_Qjet_pTcompare_optOnce.pdf}}
\subfigure[trim $m_t$+$m_W$ + all]{\includegraphics[width=0.48\textwidth]{./Figures/TTagging/multi_variable/pT_compare/Rocs_trim_mt_mw_shape_pTcompare_optOnce.pdf}}
\caption{Comparison of BDT combination of trimming + shape at different \pt using the anti-\kT R=0.8 algorithm; the tagger inputs are set to the optimum value for $\pt=1.5-1.6$ TeV.}
\label{fig:ptcomparison_trim_shape_optOnce}
\end{center}
\end{figure*}

\begin{figure*}
\begin{center}
\subfigure[prune $m_t$+$m_W$+$C_2^{(\beta=1)}$+$C_3^{(\beta=1)}$]{\includegraphics[width=0.48\textwidth]{./Figures/TTagging/multi_variable/pT_compare/Rocs_prune_mt_mw_C_pTcompare_optOnce.pdf}}
\subfigure[prune $m_t$+$m_W$+$\tau_{21}^{(\beta=1)}$+$\tau_{32}^{(\beta=1)}$]{\includegraphics[width=0.48\textwidth]{./Figures/TTagging/multi_variable/pT_compare/Rocs_prune_mt_mw_tau_pTcompare_optOnce.pdf}}
\subfigure[prune $m_t$+$m_W$ + Qjet mass volatility]{\includegraphics[width=0.48\textwidth]{./Figures/TTagging/multi_variable/pT_compare/Rocs_prune_mt_mw_Qjet_pTcompare_optOnce.pdf}}
\subfigure[prune $m_t$+$m_W$ + all]{\includegraphics[width=0.48\textwidth]{./Figures/TTagging/multi_variable/pT_compare/Rocs_prune_mt_mw_shape_pTcompare_optOnce.pdf}}
\caption{Comparison of BDT combination of pruning + shape at different \pt using the anti-\kT R=0.8 algorithm; the tagger inputs are set to the optimum value for $\pt=1.5-1.6$ TeV.}
\label{fig:ptcomparison_prune_shape_optOnce}
\end{center}
\end{figure*}

\clearpage
\subsubsection{$R$ dependence}

\begin{figure*}
\begin{center}
\subfigure[HEPTopTagger]{\includegraphics[width=0.48\textwidth]{./Figures/TTagging/multi_variable/R_compare/Rocs_HEP_Rcompare_optOnce.pdf}}
\subfigure[Johns Hopkins Tagger]{\includegraphics[width=0.48\textwidth]{./Figures/TTagging/multi_variable/R_compare/Rocs_JH_Rcompare_optOnce.pdf}}
\subfigure[JH + HEP taggers]{\includegraphics[width=0.48\textwidth]{./Figures/TTagging/multi_variable/R_compare/Rocs_HEP_JH_Rcompare_optOnce.pdf}}
\subfigure[Trimming]{\includegraphics[width=0.48\textwidth]{./Figures/TTagging/multi_variable/R_compare/Rocs_trim_mt_mw_Rcompare_optOnce.pdf}}
\subfigure[Pruning]{\includegraphics[width=0.48\textwidth]{./Figures/TTagging/multi_variable/R_compare/Rocs_prune_mt_mw_Rcompare_optOnce.pdf}}
\caption{Comparison of tagger and jet shape performance at different radius at \pt = 1.5-1.6 TeV; the tagger inputs are set to the optimum value for $R=1.2$ TeV.}
\label{fig:Rcomparison_top_optOnce}
\end{center}
\end{figure*}

\begin{figure*}
\begin{center}
\subfigure[JH+$C_2^{(\beta=1)}$+$C_3^{(\beta=1)}$]{\includegraphics[width=0.48\textwidth]{./Figures/TTagging/multi_variable/R_compare/Rocs_JH_C_Rcompare_optOnce.pdf}}
\subfigure[JH+$\tau_{21}^{(\beta=1)}$+$\tau_{32}^{(\beta=1)}$]{\includegraphics[width=0.48\textwidth]{./Figures/TTagging/multi_variable/R_compare/Rocs_JH_tau_Rcompare_optOnce.pdf}}
\subfigure[JH + Qjet mass volatility]{\includegraphics[width=0.48\textwidth]{./Figures/TTagging/multi_variable/R_compare/Rocs_JH_Qjet_Rcompare_optOnce.pdf}}
\subfigure[JH + all]{\includegraphics[width=0.48\textwidth]{./Figures/TTagging/multi_variable/R_compare/Rocs_JH_shape_Rcompare_optOnce.pdf}}
\caption{Comparison of BDT combination of JH tagger + shape at different radius at \pt = 1.5-1.6 TeV; the tagger inputs are set to the optimum value for $R=1.2$ TeV.}
\label{fig:Rcomparison_JH_shape_optOnce}
\end{center}
\end{figure*}

\begin{figure*}
\begin{center}
\subfigure[HEP+$C_2^{(\beta=1)}$+$C_3^{(\beta=1)}$]{\includegraphics[width=0.48\textwidth]{./Figures/TTagging/multi_variable/R_compare/Rocs_HEP_C_Rcompare_optOnce.pdf}}
\subfigure[HEP+$\tau_{21}^{(\beta=1)}$+$\tau_{32}^{(\beta=1)}$]{\includegraphics[width=0.48\textwidth]{./Figures/TTagging/multi_variable/R_compare/Rocs_HEP_tau_Rcompare_optOnce.pdf}}
\subfigure[HEP + Qjet mass volatility]{\includegraphics[width=0.48\textwidth]{./Figures/TTagging/multi_variable/R_compare/Rocs_HEP_Qjet_Rcompare_optOnce.pdf}}
\subfigure[HEP + all]{\includegraphics[width=0.48\textwidth]{./Figures/TTagging/multi_variable/R_compare/Rocs_HEP_shape_Rcompare_optOnce.pdf}}
\caption{Comparison of BDT combination of HEP tagger + shape at different radius at \pt = 1.5-1.6 TeV; the tagger inputs are set to the optimum value for $R=1.2$ TeV.}
\label{fig:Rcomparison_HEP_shape_optOnce}
\end{center}
\end{figure*}

\begin{figure*}
\begin{center}
\subfigure[trim $m_t$+$m_W$+$C_2^{(\beta=1)}$+$C_3^{(\beta=1)}$]{\includegraphics[width=0.48\textwidth]{./Figures/TTagging/multi_variable/R_compare/Rocs_trim_mt_mw_C_Rcompare_optOnce.pdf}}
\subfigure[trim $m_t$+$m_W$+$\tau_{21}^{(\beta=1)}$+$\tau_{32}^{(\beta=1)}$]{\includegraphics[width=0.48\textwidth]{./Figures/TTagging/multi_variable/R_compare/Rocs_trim_mt_mw_tau_Rcompare_optOnce.pdf}}
\subfigure[trim $m_t$+$m_W$ + Qjet mass volatility]{\includegraphics[width=0.48\textwidth]{./Figures/TTagging/multi_variable/R_compare/Rocs_trim_mt_mw_Qjet_Rcompare_optOnce.pdf}}
\subfigure[trim $m_t$+$m_W$ + all]{\includegraphics[width=0.48\textwidth]{./Figures/TTagging/multi_variable/R_compare/Rocs_trim_mt_mw_shape_Rcompare_optOnce.pdf}}
\caption{Comparison of BDT combination of trimming + shape at different radius at \pt = 1.5-1.6 TeV; the tagger inputs are set to the optimum value for $R=1.2$ TeV.}
\label{fig:Rcomparison_trim_shape_optOnce}
\end{center}
\end{figure*}

\begin{figure*}
\begin{center}
\subfigure[prune $m_t$+$m_W$+$C_2^{(\beta=1)}$+$C_3^{(\beta=1)}$]{\includegraphics[width=0.48\textwidth]{./Figures/TTagging/multi_variable/R_compare/Rocs_prune_mt_mw_C_Rcompare_optOnce.pdf}}
\subfigure[prune $m_t$+$m_W$+$\tau_{21}^{(\beta=1)}$+$\tau_{32}^{(\beta=1)}$]{\includegraphics[width=0.48\textwidth]{./Figures/TTagging/multi_variable/R_compare/Rocs_prune_mt_mw_tau_Rcompare_optOnce.pdf}}
\subfigure[prune $m_t$+$m_W$ + Qjet mass volatility]{\includegraphics[width=0.48\textwidth]{./Figures/TTagging/multi_variable/R_compare/Rocs_prune_mt_mw_Qjet_Rcompare_optOnce.pdf}}
\subfigure[prune $m_t$+$m_W$ + all]{\includegraphics[width=0.48\textwidth]{./Figures/TTagging/multi_variable/R_compare/Rocs_prune_mt_mw_shape_Rcompare_optOnce.pdf}}
\caption{Comparison of BDT combination of pruning + shape at different radius at \pt = 1.5-1.6 TeV; the tagger inputs are set to the optimum value for $R=1.2$ TeV.}
\label{fig:Rcomparison_prune_shape_optOnce}
\end{center}
\end{figure*}








