This report of the BOOST2012 workshop provides answers to a number of 
important questions concerning the use of jet substructure for the 
study of boosted object production at the LHC. 


We evaluated the current limitations in the description of jet substructure,
both at the analytical level and in Monte Carlo generators. 
Impressive progress is being made for the former and we expect a meaningful 
comparison to LHC 
data to be a reality soon. Two approaches - perturbative QCD and
Soft Collinear Effective Theory - to a first-principle resummation of
the jet invariant mass are producing mature results. Measurements
of the jet mass in Z+jet events are proposed, both inclusively and
exclusively in the number of jets. We hope that in
the not-too-distant future these calculations can enhance our
understanding of the internal structure in jets.

Monte Carlo predictions remain crucial to searches and measurements
employing jet substructure. We have compared the predictions of
several mainstream generators for a number of substructure observables a
and for several signal and background topologies.
While jet mass is still poorly described
by several generators, several ways of introducing the inherent
uncertainties become evident. Jet grooming reduces the spread among
Monte Carlo models, as do several alternative jet substructure 
observables.

We also studied potential experimental limitations that could check further
progress, in particular the impact of the large number of simultaneous
proton-proton interactions. We find that, even if the substructure of
large-radius jets is quite sensitive to pile-up, a combination of 
a state-of-the-art correction technique and jet grooming can effectively
restore the jet mass scale and strongly mitigate the impact on the 
jet mass resolution.

Finally, we reviewed top-tagging techniques deployed in the LHC experiments
and assessed their impact on the sensitivity to new physics. A series
of \ttbar{} resonance searches performed by ATLAS and CMS provide
clear proof of the power of techniques specifically designed for 
boosted top quarks. Through an evaluation of the impact of all
sources of systematic uncertainties, we show that further progress
can still be made with an enhanced understanding of jet substructure.
We expect to see these techniques applied in further searches involving
boosted top quarks and in measurements of the boosted top production
rate.
