



%\begin{document}

%%%%%%%%%%%%%%%%%%%%%%%%%%%%%%%%%%%%%%%%%%%%%%%%%%%%%%%%%%%%%%%%%%
%Title page
%%%%%%%%%%%%%%%%%%%%%%%%%%%%%%%%%%%%%%%%%%%%%%%%%%%%%%%%%%%%%%%%%%

%\section{BOOST 2012 Working Group: Predictions and measurements of jet substructure observables}

{\it Section prepared by the Working Group: 'Predictions and measurements of jet substructure observables', A. Davison, \underline{A. Hornig}, \underline{S. Marzani}, \underline{D.W. Miller}, G. Salam, M. Schwartz, I. Stewart, J. Thaler, \underline{N.V. Tran}, C. Vermilion,  J. Walsh) 
}

%\author{Nhan V. Tran}
%\author{David W. Miller}
%\author{Adam Davison}
%\author{Andrew Hornig}
%\author{Matthew Schwartz}
%\author{Christopher Vermilion}
%\author{Jonathan Walsh}
%\author{Simone Marzani}
%\author{Jesse Thaler}

 %ntran@fnal.gov
 %David.W.Miller@uchicago.edu
 %adamd@hep.ucl.ac.uk
 %ahornig@uw.edu
 %schwartz@physics.harvard.edu
 %verm@uw.edu
 %jwalsh@lbl.gov
 %simone.marzani@durham.ac.uk
 %jthaler@mit.edu

%\maketitle

%%%%%%%%%%%%%%%%%%%%%%%%%%%%%%%%%%%%%%%%%%%%%%%%%%%%%%%%%%%%%%%%%%
%\newpage
%\section{Motivation and Implications for Searches/Measurements of New Physics}
%\label{sec:motivation}

%%%%%%%%%%%%%%%%%%%%%%%%%%%%%%%%%%%%%%%%%%%%%%%%%%%%%%%%%%%%%%%%%%
%\newpage
%\section{Status of Substructure Measurements at ATLAS}
%\label{sec:ATLAS}

%%%%%%%%%%%%%%%%%%%%%%%%%%%%%%%%%%%%%%%%%%%%%%%%%%%%%%%%%%%%%%%%%%
%\section{Status of Substructure Measurements at CMS}
%\label{sec:CMS}

%%%%%%%%%%%%%%%%%%%%%%%%%%%%%%%%%%%%%%%%%%%%%%%%%%%%%%%%%%%%%%%%%%
%\newpage
%\section{Status of Theoretical Tools: Past, Present, and Future}
%\label{sec:theory}

The internal structure of jets has traditionally been characterized in 
jet shape measurements. A detailed introduction to the current 
theoretical understanding and of the calculations needed for 
observables that probe jet substructure is provided in last year's BOOST 
report~\cite{Altheimer:2012mn}. Here, rather 
than give a comprehensive review of the literature relevant to the myriad of 
developments, we focus on the progress made in the last year in calculations
of jet substructure at hadron colliders. Like the Tevatron experiments 
ATLAS and CMS have performed measurements of the energy flow within the 
jet~\cite{Aad:2011kq,Chatrchyan:2012mec}. Both collaborations
have moreover performed dedicated jet substructure measurements on 
large-R jets that are briefly reviewed before we introduce 
analytical calculations and summarize the status of the two main
approaches. 

\subsection{Jet Substructure Measurements by ATLAS}
\label{subsec:jsATLAS}

The first measurement of jet mass for large-radius jets ($R=1.0$, $1.2$)
and several substructure observables was performed by ATLAS on
data from the 2010 run of the LHC~\cite{ATLAS:2012am}. These early studies
include also a first measurement of the jet mass distribution for
filtered~\cite{Butterworth:2008iy} Cambridge-Aachen jets.
A number of further jet shapes were studied with the same data set
in Reference~\cite{Aad:2012meb}. 
These early studies were crucial to establish
the jet substructure response of the experiment and validate the 
Monte Carlo description of substructure. They are moreover unique, as
the impact of pile-up could be trivially avoided by selecting 
events with a single primary vertex. The results, fully corrected for
detector effects, are available for comparison to calculations.

Since then, the ATLAS experiment has performed a direct and systematic 
comparison of the performance of several grooming algorithms on 
inclusive jet samples, purified samples of 
high-$p_{T}$ $W$ bosons and top quarks, and Monte Carlo simulations of
 boosted $W$ and top-quark signal samples~\cite{Aad:2013gja}. The parameters 
of large-radius ($R=1.0$) trimmed~\cite{Krohn:2009th}, 
pruned~\cite{Ellis:2009me} and mass-drop 
filtered jet algorithms were optimized 
in the context of Standard Model measurements and new physics searches 
using multiple performance measures, including efficiency and jet mass 
resolution. 

For a subset of the jet algorithms tested, dedicated jet energy scale and 
mass scale calibrations were derived and systematic uncertainties evaluated 
for a wide range of jet transverse momenta. Relative systematic uncertainties 
were obtained by comparing ratios of track-based quantities to 
calorimeter-based quantities in the data and MC simulation. \textit{In situ} 
measurements of the mass of jets containing boosted hadronically 
decaying $W$ bosons further constrain the jet mass scale uncertainties for 
this particular class of jets to approximately $\pm1\%$.


\subsection{Jet Substructure Measurements by CMS}
\label{subsec:jsCMS}


The CMS experiment measured jet mass distributions with 
approximately 5~fb$^{-1}$ of data at a center-of-mass 
energy of $\sqrt{s}~=$ 7~TeV~\cite{Chatrchyan:2013rla}.
The measurements were performed in several $p_T$ bins and
for two processes, inclusive jet production and vector boson
production in association with jets. 
For inclusive jet production, the measurement
corresponds to the average jet mass of the highest
two $p_T$ jets. In vector boson plus jet ($V +$ jet) production
the mass of the jet with the highest $p_T$ was measured.
The measurements were performed primarily for jets clustered with the 
anti-$k_t$ algorithm with distance parameter $R=$ 0.7 (AK7).
 The mass of ungroomed, filtered, trimmed, and
pruned jets are presented in bins of pt.
Additional measurements were performed for anti-$k_t$ jets with smaller and
larger radius parameter ($R=$ 0.5, 0.8), after applying 
pruning~\cite{Ellis:2009me} and filtering~\cite{Butterworth:2008iy} 
to the jet, and for Cambridge-Aachen jets
with $R=$0.8 and $R=$ 1.2.  

The jet mass distributions are corrected for detector effects and can
be compared directly with theoretical calculations or simulation models.
The dominant systematic uncertainties are jet energy resolution effects, 
pileup, and parton shower modeling.

The study finds that, for the grooming parameters examined, the pruning 
algorithm is the most aggressive grooming algorithm, leading to the largest
average reduction of the jet mass with respect to the original jet mass.
Due to this fact, CMS also finds that the pruning algorithm reduced 
the pileup dependence of the jet mass the most of the grooming algorithms.

The jet mass distributions are compared against different simulation 
programs: \pythiasix~\cite{Sjostrand:2006za,Sjostrand:2007gs} 
(version 424, tune Z2), 
\hpp~\cite{Corcella:2000bw,Bahr:2008pv} 
(version 2.4.2, tune 23), and \pythiaeight 
(version 145, tune 4C), 
in the case of inclusive jet production.
In general the agreement between simulation and data is reasonable 
although \hpp appears to have the best agreement with the data
for more aggressive grooming algorithms. The $V + $ jet channel appears to 
have better agreement overall than the inclusive jets production 
channel which indicates 
that quark jets are modeled better in simulation.
The largest disagreement with data comes from the low jet mass region, 
which is more affected by pileup and soft QCD effects.

The jet energy scale and jet mass scale of these algorithms were validated
individually. The jet energy scale was investigated in MC simulation, and
was found to agree with the ungroomed energy scale within 3\%, which is
assigned as an additional systematic uncertainty.
The jet mass scale was investigated in a sample of boosted W bosons in a
semileptonic $\ttbar$ sample. The jet mass scale derived from the mass of the
boosted W jet agrees with MC simulation within 1\%, which is also assigned as
a systematic uncertainty.

\subsection{Analytical predictions for jet substructure}

Next-to-leading order (NLO) calculations in the strong coupling constant 
have been performed for multi-jet production, even in association with 
an electro-weak boson. This means that substructure observables, such 
as the jet mass, can be computed to NLO accuracy using 
publicly available codes~\cite{Campbell:2002tg,Nagy:2003tz}.
However, whenever multiple scales, e.g.\ a jet's transverse momentum and 
its mass, are involved in a measurement, the prediction of the observables will 
contain logarithms of ratios of these scales at each order 
in perturbation theory. These logarithms are so important for jet 
shapes that they qualitatively change the shapes as compared to fixed order.
Resummation yields a more efficient organization of the perturbative expansion 
than traditional fixed-order perturbation theory. Accurate calculations
of jet shapes are impossible without resummation. In general one can moreover
interpolate between, or {\it merge}, the resummed and fixed-order result. 

In resummation techniques the perturbative expansion of cross-sections 
for generic observables $v$ is schematically organized in the form\footnote{The 
actual form of \eq{fact} is in general rather complex. For more 
than three hard partons it involves a non-trivial matrix structure
in colour space. 
Moreover, the actual form of the constant terms $g_0(\as)$ depends 
on the flavor of the jet under consideration.}


\begin{align}
%  
& \sigma(v) =\int_0^v dv' \frac{d\sigma}{dv'} = \sum_{\begin{subarray}{c}
\text{partonic}\\ \text{configurations}\\ \delta
      \end{subarray}}
       \sigma_0^{(\delta)} g_0^{(\delta)}(\as) e^{\beta},  
\label{eq:fact}
       \\
      & \beta = {Lg_1^{(\delta)}(\alpha_s L)+g_2^{(\delta)}(\alpha_s L) +\as g_3^{(\delta)}(\alpha_s L) +\dots}
\label{eqresum}
\end{align}
where $\sigma_0=\sum \sigma_0^{(\delta)}$ is the corresponding Born 
cross-section and
$L = \ln v$ is a logarithm of the observable in question\footnote{In the following we concentrate on the case of jet masses with a cut on the jet $p_T$. 
In this case $L=\ln m_J^2/p_T^2$ and $\sigma(v)$ in Eq.~(\ref{eq:fact}) 
is the integrated (cumulative) distribution for $m_J^2< v p_T^2$.}. 

 The notation used in traditional 
fixed-order perturbation theory refers to the lowest-order calculation as 
leading order (LO) and higher-order calculations as next-to-leading order 
(NLO), next-to-next to leading order (NNLO), and so on (with N${^n}$LO 
referring to the $\mathcal{O}(\alpha_s^n)$ correction to the LO result). 
When organized instead in resummed perturbation theory as in \eq{fact}, 
the lowest order, in which only the function $g_1^{(\delta)}$ is retained, 
is referred to as leading-log (LL) approximation. 
Similarly, the inclusion of all $g_i^{(\delta)}$ with $1\leq i\leq k+1$ 
and of $g_0$ up to order $\as^{k-1}$ gives the next$^k$-to-leading log 
approximation to $\ln \sigma$; this corresponds to the resummation of 
all the contributions of the form $\as^n \ln^m N$
with $2(n-k)+1\le m\le 2n$ in the cross section $\sigma$. This can be 
extended to $2(n-k)\le m\le
2n$ by also including the order $\as^k$ contribution to $g_0^{(\delta)}$ . 


Typical Monte Carlo event generators such as \pythia, \hpp and
Sherpa~\cite{Gleisberg:2008ta} are correct at LL. 
NLL accuracy has also been achieved for some specific observables, 
but it is difficult to say whether this can be generally obtained. 
Analytic calculations provide a way of obtaining 
precise calculations for jet substructure.
Multiple observables have been resummed (most often at least to NLL 
but not uncommonly to NNLL and as high as  NNNLL accuracy for a few cases) 
and others are actively being studied and calculated in the theory community.

Often for observables of experimental interest, non-global logarithms (NGLs) 
arise~\cite{Dasgupta:2001sh}, in particular whenever a hard boundary in 
phase-space is present (such as a rapidity cut or a geometrical jet boundary). 
These effects enter at NLL level and therefore modify the structure of the 
function $g_2^{(\delta)}$ in \eq{fact}. Until very recently~\cite{Hatta:2013iba}, 
the resummation of NGLs was confined to the limit of large number of 
colours $N_C$~\cite{Dasgupta:2001sh,Dasgupta:2002bw,Banfi:2002hw}. 

Moreover, we should stress that another class of contributions, usually 
referred to as clustering logarithms, affects the $g_2^{(\delta)}$ series 
of \eq{fact} if an algorithm other than anti-$k_t$ is used to define the 
jets~\cite{Banfi:2005gj,Delenda:2006nf}. The analytic structure of these 
clustering effects has been recently explored in 
Ref.~\cite{Kelley:2012zs,Delenda:2012mm} for the case of Cambridge-Aachen 
and $k_t$ algorithms.
 
Furthermore, recent studies have shown that strict collinear factorization is violated if the observable considered is not sufficiently inclusive~\cite{Catani:2011st,Forshaw:2012bi}. As a consequence, coherence-violating (or super-leading) logarithms appear, which further complicate the resummation of certain observables. These contributions affect, for example, non-global dijets observables~\cite{Kyrieleis:2005dt,Forshaw:2008cq} but also some classes of global event shapes~\cite{Banfi:2010xy}.


Of course, to fully compare 
to data one needs to incorporate the effects of hadronization and 
multi-particle interactions (MPI). Progress on this front has also 
been made, both in purely analytical approaches (especially for 
hadronization effects~\cite{Mateu:2012nk}) and in interfacing analytical 
results with parton showers that incorporate these effects.
%%%%%CITATIONS

The two main active approaches to resummation are referred to as traditional 
perturbative QCD resummation (pQCD) and Soft Collinear Effective Theory (SCET).
They describe the same physical effects, which are captured by the 
Eqs.~\ref{eq:fact} and~\ref{eqresum}. However, the techniques employed in pQCD and SCET 
approaches often differ. Calculations in pQCD exploit factorization 
and exponentiation properties of QCD matrix elements and of the 
phase-space associated to the 
observable at hand, in the soft or collinear limits. The SCET approach 
is based on factorization at the operator level and exploits
the renormalization group to resum the logarithms. The two approaches also 
adopt different philosophies for the treatment of NGLs. A more detailed 
description of these differences is given in the next Sections.

 

%To date, most of the work done in the context of pQCD has concentrated on 
%producing NLL resummed predictions, with NGLs treated in the large-$N_C$ 
%limit. 
%On the other hand, the SCET community has spent more time engineering 
%observables for which NGLs have a minimal (or null) impact.
%Furthermore, calculations in SCET are most easily performed for observables 
%with a fixed number of jets (i.e.\ exclusive). We turn to a more detailed
%description of these differences in the the next Sections.









%%%%%%%%%%%%%%%%%%%%%%%%%%%%%%%%%%%%%%%%%%%%%%%%%%%%%%%%%%%%%%%%%%
\subsection{Resummation in pQCD}
\label{subsec:pQCD}

Jet mass was calculated in pQCD in~\cite{Li:2011hy}. A more 
extensive study can be found in Ref.~\cite{Dasgupta:2012hg} where the jet mass 
distribution for Z+jet and inclusive jet production, with jets defined with the 
anti-$k_t$ algorithm, were calculated at NLL accuracy and matched to LO.
%
In particular, for the Z+jet case, the jet mass distribution of the 
highest $p_T$ jet was calculated whereas for inclusive jet production, 
essentially the average of jet mass distributions of the two highest 
$p_T$ jets was calculated.
For the Z+jet case, one has to consider soft-wide angle emissions from a
three hard parton ensemble, consisting of the incoming partons and the
outgoing hard parton.
For three or fewer partons, the colour structure is trivial. Dijet production 
on the other hand involves an ensemble of four hard partons and the 
consequent soft wide-angle radiation has a non-trivial colour matrix 
structure. The rank of these matrices grows quickly with the number 
of hard partons, making the calculations for multi jet final states a 
formidable challenge~\footnote{The colour structure of soft gluon 
resummation in a multi jet environment has been studied 
in~\cite{Sjodahl:2008fz,Sjodahl:2009wx} and resummed calculations 
for the case of five hard partons in the context of jet production 
with a central jet veto can be found in 
Refs~\cite{Kyrieleis:2005dt,Forshaw:2008cq,Forshaw:2006fk,DuranDelgado:2011tp}}.

%% tot hier

The jet mass is a non-global observable and NGLs of $m_J/p_T$ for jets 
with transverse momentum $p_T$ are induced. Their effect was 
approximated using an analytic formula with coefficients 
fit to a Monte Carlo simulation valid in the large $N_C$ limit, obtained 
by means of a dipole evolution code~\cite{Dasgupta:2001sh}. It was found 
that in inclusive calculations\footnote{We refer to inclusive calculations
if no requirements were made on the number of additional jets in the 
selection of the event.} 
the effects of both the soft wide-angle radiation and the NGLs, both 
of which affect the $g_2^{(\delta)}$ series in \eq{fact}, play a relevant 
role even at relatively small values of jet radius such as $R=0.6$ and 
hence in general cannot be neglected

A restriction on the number of additional jets could be implemented, 
for instance, by vetoing additional jets with $p_T>p_T^\text{cut}$. 
The presence of a jet veto modifies the calculation in several ways. 
First of all, it affects the argument of the non-global logarithms: 
$\ln^n (m_J^2/p_T^2) \to \ln^n (m_J^2/(p_T p_T^{cut}))$. Thus $p_T^\text{cut}$ 
could be in principle used to tame the effect of NGLs. 
However, if the veto scale is chosen such that $p_T^\text{cut}\ll p_T$, 
logarithms of this ratio must be also resummed. Depending on the specific 
details of the definition of the observable, this further resummation can 
be affected by a new class of NGLs~\cite{Banfi:2010pa,KhelifaKerfa:2011zu}.


An obstacle to inclusive predictions in the number of jets is that the constant 
term $g_0^{(\delta)}$ in \eq{fact} receives contributions from higher jet 
topologies that are not related to any Born configurations. For instance, 
the jet mass in the Z+jet process would receive contributions from Z+2jet 
configurations, which are clearly absent in the exclusive case. The full 
determination of the constant term to $\mathcal{O}(\alpha_s)$ and the 
matching to NLO is ongoing.


%%%%%%%%%%%%%%%%%%%%%%%%%%%%%%%%%%%%%%%%%%%%%%%%%%%%%%%%%%%%%%%%%%
\subsection{Resummation in SCET}
\label{subsec:SCET}

There have been several recent papers in SCET directly related to 
substructure in hadron collisions\footnote{We consider here only 
research made publicly available at the time of BOOST 2012 or soon after.}. 
Ref.~\cite{Chien:2012ur} discusses the resummation of jet mass by expanding
around the threshold limit, where (nearly) all of the energy goes into the
final state jets. Expanding around the threshold limit has proven effective 
for other observables, see Ref.~\cite{Laenen:1998qw} and references in
Ref.~\cite{Chien:2012ur}.
The large logarithms for jet mass are mainly due
to collinear emission within the jet and soft emission from the recoiling
jet and the beam. These same logarithms are present near threshold and the
threshold limit automatically prevents additional jets from being 
relevant, simplifying the calculation. The study in Ref.~\cite{Chien:2012ur}
performs resummation at the NNLL level, but does not include NGLs. Instead,
their effect is estimated and found to be subdominant in the peak region,
where other effects, such as nonperturvative corrections, are comparable. 
Thus NGLs could be safely ignored where the calculation was most accurate. 

An alternative approach using SCET is found in Ref.~\cite{Jouttenus:2013hs}. 
Beam functions are used to contain the collinear radiation from the beam
remnants. The jet mass distribution 
in Higgs+1jet events is studied via the factorization formula for 1-jettiness, 
that is calculated to NNLL accuracy. Using 1-jettiness, the jet boundaries 
are defined by the distance measure used in 1-jettiness itself, instead of 
a more commonly employed jet algorithm, although generalizations to
arbitrary jet algorithms are possible.

For a single jet in hadron collisions, 1-jettiness can be used as a 
means to separate the in-jet and out-of-jet radiation (see for 
a review the BOOST2011 report\cite{Altheimer:2012mn}). The observable 
studied in Ref.~\cite{Jouttenus:2013hs} is 
separately differential in the jet mass and the beam thrust. 
The in-jet component 
is related to the jet mass, and can be converted directly up to 
corrections that become negligible for higher $p_T$ (up to about 3\% 
for $p_T = 300$~GeV in the peak of the distribution of the in-jet 
contribution to 1-jettiness which is smaller than NNLL uncertainties). 
The beam thrust\footnote{The resummation of beam thrust
is analogous to that of thrust in $e^+e^-$ 
collisions~\cite{Berger:2010xi,Stewart:2010pd,Stewart:2009yx}.} 
is a measure of the out-of-jet contributions, equivalent to a
rapidity-weighted veto scale $p_{\rm cut}$ on extra jets. The calculation
can be made exclusive in the number of jets by making the out-of-jet 
contributions small. 
Where Ref.~\cite{Chien:2012ur} ensures a fixed number
of jets by expanding around the threshold limit, Ref.~\cite{Jouttenus:2013hs}
includes an explicit jet veto scale.


Exclusive calculations in the number of jets avoid some of the 
issues mentioned in \subsec{pQCD}. An important property of 
1-jettiness is that, when considering the sum of the in- and out-of-jet 
contributions, no NGLs are present, and when considering these 
contributions separately, only the ratio $p_{\rm cut} / m_J$ of these two 
scales is non-global. A smart choice of the veto scale may then allow
to minimize the NGL and make the resummation unnecessary.
This corresponds to the NGLs discussed in 
\subsec{pQCD} that are induced in going from the inclusive to the 
exclusive case. These are the only NGLs present; 
the additional NGLs of the measured jet $p_T$ to their mass discussed 
for the observable of \subsec{pQCD} are absent in this case.
By using an exclusive observable, with an explicit veto scale, NGLs
are controlled. For comparison with inclusive jet mass measurements,
such as those discussed in Sections~\ref{subsec:jsATLAS} and~\ref{subsec:jsCMS},
the uncertainty associated with the veto scale can be estimated in
a similar fashion as the NGL estimate in Ref.~\cite{Chien:2012ur}.

It was argued in Ref.~\cite{Jouttenus:2013hs} that the NGLs induced 
by imposing a veto on both the $p_T$ and jet mass are smaller 
than the resummable logarithms of the measured jets over a range of veto 
scales. In contrast,
in the inclusive case the corresponding $p_T$ value that appears in the 
NGLs is of the order of the measured jet $p_T$ (since all values less 
than this are allowed), making it a large scale and the NGLs as large 
as other logarithms. For a fixed veto cut, it was argued that the effect 
of these NGLs (at least of those that enter at the first non-trivial 
order, $\mathcal{O}(\alpha_s^2)$), can be considered small enough to 
justify avoiding resummation for a calculation up to NNLL accuracy 
for $1/\sqrt{8} < m_J^{\rm cut}/p_{\rm cut} < \sqrt{8}$ (cf. 
Ref.~\cite{Hornig:2011iu}) in the peak region where a majority 
of events lie. It is also worth noting that the effect of normalizing 
the distribution by the total rate up to a maximum $m_J^{\rm cut}$ and 
$p_{\rm cut}$ has several advantages and in particular has a smaller 
perturbative uncertainty than the unnormalized distribution, in 
addition to having smaller experimental uncertainties.



We also note that while jet mass is now the most well-understood 
substructure observable, it is also clearly much simpler than the 
more complicated techniques often employed by experimentalists in 
boosted studies. There has also been progress in understanding 
more complicated measurements using SCET, and in particular a 
calculation of the signal distribution in $H \to b \bar{b}$ was 
performed in Ref.~\cite{Feige:2012vc}. While it is probably fair 
to say that our theoretical understanding (or at least the numerical 
accuracy) of such measurements are currently not at the same level 
as that of the jet mass, this is a nice demonstration that reasonably 
accurate calculations of realistic substructure measurements can be 
performed with the current technologies and that it is not 
unreasonable to expect related studies in the near future.

%%%%%%%%%%%%%%%%%%%%%%%%%%%%%%%%%%%%%%%%%%%%%%%%%%%%%%%%%%%%%%%%%%
\subsection{Discussion and recommendations for further substructure measurements}
\label{subsec:rec}

We have presented a status report for the two main approaches to 
the resummation of jet substructure observables, with a 
focus on their potential to 
predict the jet invariant mass at hadron colliders. In both approaches
recent work has shown important progress

We hope that providing predictions beyond the accuracy of 
parton showers may help both discovery and 
measurement. Beyond the scope of improving our understanding of QCD, 
gaining intuition for which treatments work best is an important step 
towards adopting such predictions as an alternative to parton showers. 
Non-perturbative corrections like 
hadronization are more complicated at the LHC due to the increased colour 
correlations. Entirely new perturbative and semi-perturbative 
effects such as multiple-particle interactions appear. 
Monte Carlo simulations suggest that these have a significant impact.

The treatments of non-perturbative corrections and NGLs are 
often different in pQCD and SCET~\footnote{We have focused on differences in our
discussion, but typically both communities have the option to adopt the 
treatments commonly employed in the other community. That is, the treatments 
typically utilized are not features inherent to the approach.} and this
leads to slight differences in which measurements are best suited for
comparison to predictions. The first target for the next year should 
be a phenomenological study of the jet mass distribution in Z+jet, for 
which we encourage ATLAS and CMS measurements. Ideally, since the QCD and SCET 
literature have emphasized a difference in preference for inclusive
or exclusive measurements (in the number of jets),
both should be measured to help our understanding of the two 
techniques.

The importance of  boosted-object taggers in searches for new physics 
will increase strongly in the near future in view of the 
higher-energy and higher-luminosity LHC runs. However, the theoretical 
understanding of these tools is in its infancy. Analytic calculations 
must be performed in order to understand the properties of the different 
taggers and establish which theoretical approaches 
(MC, resummation or even fixed order) are needed to accurately compute 
these kind of observables~\footnote{Before the completion of this 
manuscript, two papers appeared~\cite{Dasgupta:2013tia,Dasgupta:2013via} 
which perform analytic resummed calculations for boosted-object methods, 
such as trimming, pruning and mass drop, and energy correlations were 
computed and used for quark and gluon discrimination in 
Ref.~\cite{Larkoski:2013eya}.}.

%%%%%%%%%%%%%%%%%%%%%%%%%%%%%%%%%%%%%%%%%%%%%%%%%%%%%%%%%%%%%%%%%%




