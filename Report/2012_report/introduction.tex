
With a centre-of-mass energy of 7~\tev{} in 2010 and 2011 and 
of 8~\tev{} in 2012 the LHC has pushed the energy frontier
well into the~\tev{} regime. Another leap in energy is expected 
with the start of the second phase of operation in 2014, when the 
centre-of-mass energy is to be increased to 13-14~\tev{}.
For the first time experiments produce large samples of $W$ and $Z$ 
bosons and top quarks with a transverse momentum $p_T$ that considerably 
exceeds their rest mass $m$ ($p_T \gg m$). The same is true also for the Higgs 
boson and, possibly, for as yet unknown particles with masses near
the electroweak scale.  
In this new kinematic regime, well-known particles are observed in unfamiliar 
ways. {\it Classical} reconstruction algorithms that rely on a one-to-one
jet-to-parton assignment are often inadequate,
in particular for hadronic decays of such boosted objects.

A suite of techniques has been developed to fully exploit the opportunities 
offered by boosted objects at the LHC. Jets are reconstructed 
with a much larger radius parameter to capture the energy of the complete 
(hadronic) decay in a single jet. The internal structure of these 
{\it fat} jets is a key signature to identify boosted objects among the 
abundant jet production at the LHC. Many searches use a 
variety of recently proposed substructure observables.
Jet grooming techniques\footnote{We refer to three related techniques as jet
grooming: filtering~~\cite{Butterworth:2008iy}, 
trimming~\cite{Krohn:2009th} and 
pruning~\cite{Ellis:2009me}. Unless stated otherwise, all studies
in this paper of these techniques apply a common set of parameters that
is widely used in the community.} 
improve the resolution of jet substructure 
measurements, help to reject background, 
and increase the resilience to the impact of 
multiple proton-proton interactions.

In July 2012 IFIC Valencia organized the 2012 edition~\cite{boost12} 
of the BOOST series of workshops, the main forum for the physics of 
boosted objects and jet substructure\footnote{Previous BOOST workshops
took place at SLAC (2009, ~\cite{boost09}), 
Oxford University (2010,~\cite{boost10}) and Princeton 
University University (2011,~\cite{boost11}). BOOST2013~\cite{boost13} 
was organized by the University of Arizona from August 12$^{th}$ to 16$^{th}$.}.
Working groups formed during the 2010 and 2011 workshops prepared 
reports~\cite{Abdesselam:2010pt,Altheimer:2012mn} that provide an
overview of the state of the field and an entry point to the 
now quite extensive literature and present new material prepared 
by participants. 
In this paper we present the report of the working groups set up 
during BOOST2012. Each contribution addresses an important aspect 
of jet substructure as a tool for the study of boosted objects at the LHC.

A good understanding of jet substructure is a prerequisite to further
progress. Predictions of jet substructure based on first-principle,
analytical calculations may provide a more precise description
of jet substructure and allow deeper insight. However, resummation 
of the leading logarithms in this case is notoriously difficult and the
predictions may be subject to considerable uncertainties.
In fact, one might ask:
\begin{itemize}
\item Can jet substructure be predicted by first-principle QCD calculations
and compared to data in a meaningful way? 
\end{itemize}  
The findings of the working group that was set up to evaluate 
the limitations and potential of the most popular approaches are 
presented in Section~\ref{sec:qcd}.

While progress toward analytical predictions continues, searches for 
boosted objects that employ jet substructure rely on 
the predictions of mainstream Monte Carlo models. It is therefore
vital to answer this question:
\begin{itemize}
\item How accurately is jet substructure described by state-of-the-art Monte Carlo tools?
\end{itemize}
The BOOST2010 report~\cite{Abdesselam:2010pt} provided a 
partial answer, based on pre-LHC tunes of several popular leading-order 
generators. After the valuable experience gained in the first three years of 
operation of the LHC, it seems appropriate to revisit this question
 in Section~\ref{sec:mc}. 

A further potential limitation to the performance of jet substructure 
is the level to which the detector response can be understood and modelled. 
Again, the first years of LHC operation have provided valuable experience 
on how well different techniques work in a realistic experimental environment. 
In particular, the impact of multiple proton-proton 
interactions (pile-up) on substructure measurement has been evaluated
exhaustively and mitigation schemes have been developed. 
Anticipating a sharp increase in the pile-up activity in future operating
scenarios of the LHC, one might worry that in the future the detector 
performance might be degraded considerably for the sensitive 
substructure analyses. A third working group was therefore given the 
following charge:
\begin{itemize}
\item  How does the impact of additional proton proton collisions 
limit jet substructure performance at the LHC, now and in future 
operating scenarios?  
\end{itemize}
Section~\ref{sec:pileup1} presents the contributions 
regarding jet reconstruction performance under 
extreme contributions, with up to 200 additional proton-proton collisions
in each bunch crossing. We present the prospects for fake jet
rates and the impact of pile-up on jet mass measurements 
under these conditions.

In the first years of operation of the LHC several groups in ATLAS and CMS
have deployed techniques specifically developed for the study of 
boosted objects in several analyses. Jet substructure has become 
an important tool in many searches for evidence for new physics. 
In Section~\ref{sec:top} we present the lessons learnt in several
studies of boosted top quark production that have been the first
to apply these techniques and answer the following question: 
\begin{itemize}
\item How powerful is jet substructure in studies of boosted top production, and how can it be made even more powerful?
\end{itemize}
%The next two Sections
%of the report thus answer the following questions:
%\begin{itemize}
%\item Section~\ref{sec:top} What is the potential of the study of boosted top quarks?
%\item Section~\ref{sec:higgs} What are the prospects for observation for observation of the Higgs-like boson in the $H \rightarrow b \bar{b}$ channel and measurement of the b-quark Yukawa coupling?
%\end{itemize}

%%We hope that this report may prove a useful portal to the abundant literature
%%on boosted objects.
We hope that the answers to the above questions prepared by the working
groups may shed some light on this rapidly evolving field. 

